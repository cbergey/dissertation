% ---- ETD Document Class and Useful Packages ---- %
% v1.3.0 released April 12, 2023.
% https://github.com/k4rtik/uchicago-dissertation
\documentclass{ucetd}

\usepackage[T1]{fontenc}
\usepackage{subcaption,graphicx}
\usepackage{natbib}
\usepackage{mathtools}  % loads amsmath
\usepackage{amssymb}    % loads amsfonts
\usepackage{amsthm}

%% Use these commands to set biographic information for the title page:
\newcommand{\thesistitle}{Remarkable features: Using descriptive contrast to express and infer typicality}
\newcommand{\thesisauthor}{Claire Augusta Bergey}
\department{Psychology}
\division{Social Sciences}
\degree{Doctor of Philosophy}
\date{August 2023}

\title{\thesistitle}
\author{\thesisauthor}

%% Use these commands to set a dedication and epigraph text
\dedication{Dedication Text}
\epigraph{Epigraph Text}

\usepackage{doi}
\usepackage{xurl}
\hypersetup{bookmarksnumbered,
            linktoc=all,
            pdftitle={\thesistitle},
            pdfauthor={\thesisauthor},
            pdfsubject={},                                % Add subject/description
            % pdfkeywords={keyword1, keyword2, keyword3}, % Uncomment and revise keywords
            pdfborder={0 0 0}}
% See https://github.com/k4rtik/uchicago-dissertation/issues/1
\makeatletter
\let\ORG@hyper@linkstart\hyper@linkstart
\protected\def\hyper@linkstart#1#2{%
  \lowercase{\ORG@hyper@linkstart{#1}{#2}}}
\makeatother
\newlength{\cslhangindent}
\setlength{\cslhangindent}{1.5em}
\newenvironment{cslreferences}%
{\setlength{\parindent}{0pt}%
\everypar{\setlength{\hangindent}{\cslhangindent}}\ignorespaces}%
{\par}
\begin{document}
%% Basic setup commands
% If you don't want a title page comment out the next line and uncomment the line after it:
\maketitle
%\omittitle

% These lines can be commented out to disable the copyright/dedication/epigraph pages
\makecopyright
%\makededication
%\makeepigraph


%% Make the various tables of contents
\tableofcontents
\phantomsection
\newpage
\phantomsection
\listoffigures
\newpage
\phantomsection
\listoftables
\newpage

\acknowledgments
% Enter Acknowledgements here
Each chapter in this dissertation represents collaborative work. The collaborators on each chapter are: Chapter 1, Benjamin Morris and Dan Yurovsky; Chapter 2, Dan Yurovsky; Chapter 3, Jordyn Martin. None of this work could have happened without Ben, Dan, and Jordyn, and anything this dissertation gets right can be credited to their intellectual contributions to these chapters and to my life.

Thank you to my committee, Susan Goldin-Meadow, Marisa Casillas, Howard Nusbaum, and Dan Yurovsky, for all they have given me: frameworks to think in, an intellectual community to be a part of, and guidance to follow (or not). There are few gifts better than a new way to see the world.

Susan, you have been a deep intellectual influence on me; I count myself lucky that you took me in. Dan, you gave me my theoretical foundations—I hope to carry them forward. Marisa, you shook up my thinking in ways just beginning to take shape in my work. Simon DeDeo, I never know where our meetings will go. All of you gather people together to think, and I'm thankful to have been one of them.

I am deeply grateful to people who have given feedback on this work along the way: Ming Xiang, Benjamin Morris, Ashley Leung, Michael C. Frank, Judith Degen, Stephan Meylan, Robert Hawkins, and Ruthe Foushee. Thanks also to reviewers and attendees of Experiments in Linguistic Meaning, the meeting of the Cognitive Science Society, the Midwestern Cognitive Science Conference, the Dubrovnik Conference on Cognitive Science, the meeting of the Society for Research in Child Development, and the journal Cognition.

Many people made up my life while I wrote this—too many to mention. I'm pretty sure I can't think without talking to people, and you all are my best interlocutors. A few choice words for choice people:

To Jean and Barry, for the love of thinking and art—you made me and made me who I am; To Margaret, your pride means the world; To Matthew, it's a pleasure being siblings with you, wouldn't have it any other way; 

To Julia, you are my best critic—you see what the best of something can be—and always by my side; To Rosemary, for relentlessly collaborating with me, you help me keep hold of my self;

To Scott, who has seen it all happen, for talking with me all this time; To Russell, here's to us; To Jane and Erica, my oldest friends;

To Ben, for bearing with me in research and in friendship; To Hannah, for being strange and remembering why we like each other; To Ya\u{g}mur, for challenging and care in equal measure; 

To Casey, for learning to teach together, and for the jokes; To Ashley, we did it; To Jimmy, for not taking it all too seriously;

To Ruthe, for telling me you can do whatever you want and being right; To Ivan, for bringing people together; To Mike, for arguing with me and sitting with me.

I love you all. None of you will ever get rid of me.

\abstract
% Enter Abstract here
We mention what is remarkable while letting the unremarkable go unsaid. Thus, while language can tell us a lot about the world, it does not veridically reflect the world: people are more likely to talk about atypical features (e.g., ``purple carrot'') than typical features (e.g., ``[orange] carrot''). In this dissertation, I characterize how people selectively describe the features of things and examine the implications of this selective description for how children and adults learn from language. In Chapter 1, I show that adults speaking to other adults, caregivers speaking to children, and children themselves tend to mention the atypical more than the typical features of concrete things. Language is structured to emphasize what is atypical—so how can one learn about what things are typically like from language? In this chapter I also show that distributional semantics models that use word co-occurrence to derive word meaning (word2vec) do not capture the typicality of adjective–noun pairs well. I also examine the performance of two more sophisticated language models (BERT and GPT-3); these models have input unlike what children have access to, but provide useful bounds on the typicality information learnable from applying simple training objectives to language alone. However, people can learn about typicality in other ways: in Chapter 2, I show that people infer that mentioned features are atypical. That is, when a novel object is called a ``purple toma,'' adults infer that tomas are less commonly purple in general. This inference is captured by a model in the Rational Speech Act framework that posits that listeners reason about speakers' communicative goals. In Chapter 3, I ask: do children themselves infer that mentioned features are atypical? I find preliminary evidence that 5- to 6-year-old children who reliably respond on our typicality measure tend toward making contrastive rather than associative inferences; further work is necessary to confirm this finding and test younger children's contrastive inferences. Overall, this dissertation examines how language does not directly reflect the world, but selectively picks out remarkable facets of it, and what this implies for how adults, children, and language models learn.

\mainmatter
% Main body of text follows
\hypertarget{introduction}{%
\chapter*{Introduction}\label{introduction}}
\addcontentsline{toc}{chapter}{Introduction}

An utterance can say much more about the world than its literal
interpretation might suggest. For instance, if you hear a colleague say
``We should hire a female professor,'' you might infer something about
the speaker's goals, the makeup of a department, or even the biases of a
field---none of which is literally stated. These inferences depend on
recognition that a speaker's intended meaning can differ from the
literal meaning of their utterance, and the process of deriving this
intended meaning is called \emph{pragmatics}. Frameworks for
understanding pragmatic inference posit that speakers tend to follow
general principles of conversation---for instance, that they tend to be
relevant, brief, and otherwise helpfully informative (Grice 1975;
Sperber and Wilson 1986; Clark 1990). When a speaker deviates from these
principles, a listener can reason about the alternative utterances the
speaker might have said and infer some intended meaning that goes beyond
the literal meaning of their utterance.

Beyond enriching the interpretation of utterances whose literal meaning
is known, pragmatic inference is a potentially powerful mechanism for
learning about new words and concepts. People can learn the meanings of
words by tracking associations between word use and present objects
alone (Yu and Smith 2007), but reasoning about a speaker's intended
meaning---not just relating the words they say to objects in the
environment---may support more rapid and more accurate learning (Frank,
Goodman, and Tenenbaum 2009). For example, Akhtar, Carpenter, and
Tomasello (1996) showed that young children can infer the meaning of a
new word by using the principle that people tend to remark on things
that are new and interesting to them. In this study, an experimenter
leaves the room and a new toy emerges in her absence; once she comes
back, the toy is familiar to the child but not to the experimenter. When
she uses a novel name, ``gazzer,'' the child can infer that the word
refers to the toy that is novel to the experimenter, and not to other
toys the experimenter had already seen. Experiments with adults show
that they too can use general principles of informativeness to infer a
novel referent's name (Frank and Goodman 2014).

One potential pragmatic tool for learning about referents is contrastive
inference from description. To the extent that communicators strive to
be minimal and informative, description should discriminate between the
referent and some relevant contrasting set. This contrastive inference
is fairly obvious from some types of description, such as some
postnominal modifiers: ``The door with the lock'' clearly implies a
contrasting door without one (Ni 1996). The degree of contrast implied
by more common descriptive forms, such as prenominal adjectives in
English, is less clear: speakers do not always use prenominal adjectives
minimally, often describing more than is needed to establish reference
(Engelhardt, Barış Demiral, and Ferreira 2011; Mangold and Pobel 1988;
Pechmann 1989). Nevertheless, Sedivy et al. (1999) showed that people
can use these inferences to resolve referential ambiguity in familiar
contexts. When asked to ``Pick up the tall cup,'' people directed their
attention more quickly to the target when a short cup was present, and
did so in the period before they heard the word ``cup.'' Because the
speaker would not have needed to specify ``tall'' unless it was
informative, listeners were able to use the adjective to direct their
attention to a tall object with a shorter counterpart. Subsequent work
using similar tasks has corroborated that people can use contrastive
inferences to direct their attention among familiar referents (Sedivy
2003; Aparicio, Xiang, and Kennedy 2016; Ryskin, Kurumada, and
Brown-Schmidt 2019).

But what if you didn't know the meaning of the key words in someone's
utterance---could you use the same kind of contrastive inferences to
learn about new words and categories? Suppose a friend asks you to
``Pass the tall dax.'' Intuitively, your friend must have said the word
``tall'' for a reason. One possibility is that your friend wants to
distinguish the dax they want from another dax they do not. In this
case, you might look around the room for two similar things that vary in
height, and hand the taller one to them. If, alternatively, you only see
one object around whose name you don't know, you might draw a different
inference: this dax might be a particularly tall dax. In this case, you
might think your friend used the word ``tall'' for a different
reason--not to distinguish the dax they want from other daxes around
you, but to distinguish the dax they want from other daxes in the world.
This would be consistent with data from production studies, in which
people tend to describe atypical features more than they describe
typical ones (Mitchell, Reiter, and Deemter 2013; Westerbeek, Koolen,
and Maes 2015; Rubio-Fernández 2016). For instance, people almost always
say ``blue banana'' to refer to a blue banana, but almost never say
``yellow banana'' to refer to a yellow one. In each of these
cases---when distinguishing the dax from other referents nearby, or from
daxes in general---you would have used a pragmatic inference to learn
something new about the category of daxes.

This dissertation will explore the ways in which people can learn about
new words and categories from contrastive inference, with an eye toward
understanding how contrastive inference could help children learn about
language and the world it describes. To set the stage for understanding
how listeners use contrastive inference, we first need to establish that
speakers use adjectives in informative ways.

In Chapter 1, we investigate whether people tend to use adjectives to
remark on the atypical features (e.g., ``the purple carrot'') rather
than the typical features (e.g., ``the {[}orange{]} carrot'') of things.
In a corpus study of caregivers' speech, we show that caregivers tend to
mention atypical rather than typical features of things when speaking to
their children. We also show that adults speaking to other adults in
naturalistic contexts tend to remark on atypical features rather than
typical ones, extending findings from reference game tasks in the lab
(Mitchell, Reiter, and Deemter 2013; Westerbeek, Koolen, and Maes 2015;
Rubio-Fernández 2016). Finally, we show that children's own speech
mentions atypical more than typical features, and discuss the
implications of this finding for our understanding of children's
pragmatic competence.

Given that speech emphasizes atypical features, learning about
typicality from language may not be straightforward. In an analysis
using language models, we examine whether it is possible to learn about
the typical features of things using the statistical patterns within
language alone. To do this, we examine whether three language models
(word2vec, BERT, and GPT-3) capture typicality relationships between
nouns and adjectives. We find that word2vec and BERT do not represent
typicality well: likely because they use associative methods to
represent word meaning and their input tends to highlight atypical
features, these models represent the relationship between nouns and
adjectives poorly. However, GPT-3, a larger model trained on much more
language than children have access to, captures noun-adjective
typicality fairly accurately. We discuss implications for children's
word learning as well as for language modeling.

In Chapter 2, we establish that adults can use contrastive inferences to
learn about a new category's feature distribution. People use adjectives
for multiple communicative purposes: in some cases, an adjective is
needed to pick out one object among others in the immediate environment
(e.g., ``the tall cup'' contrasts with a nearby shorter cup, but is not
especially tall); in others, it marks atypicality (e.g., ``the tall
cup'' is taller than most cups in general). In this chapter, we use two
experiments with adults to show that people can use contrastive
inferences to learn about a new category's feature distribution. People
observe instances of novel categories and hear them described (e.g.,
``Pass me the {[}green{]} toma''), and then judge the prevalence of the
relevant feature (e.g., how common it is for tomas to be green). People
infer that mentioned features are less prevalent than unmentioned ones,
and do so even when the feature had to be mentioned to establish
reference. We use a model in the Rational Speech Act (RSA) framework to
capture people's judgments, finding that their judgments are consistent
with graded consideration of both reference and conveying typicality as
purposes of using an adjective.

In Chapter 3, we present a preliminary study of children's own
contrastive inferences. We test whether children infer that, for
example, mentioning that a certain object is tall, blue or spotted
implies that other group members are less likely to have those features.
However, testing children in this kind of task presents a key
difficulty: young children often struggle with the kinds of scales we
use to ask adults about typicality. Our study therefore has two goals:
both to examine whether 5- to 6-year-old children can sensibly report
typicality on a scale from \emph{few} to \emph{almost all}, and to
gather preliminary evidence about their contrastive inferences. We find
that though about half of children in this age range struggle with this
measure, children who do understand the measure make judgments in the
direction of contrastive inference. We discuss the implications of this
kind of inference for children's learning given the descriptions they
hear from caregivers, and the potential unintended consequences of
remarking on individuals' traits for children's learning.

\hypertarget{people-talk-more-about-atypical-than-typical-features-of-things}{%
\chapter{People talk more about atypical than typical features of
things}\label{people-talk-more-about-atypical-than-typical-features-of-things}}

Children learn a tremendous amount about the structure of the world
around them in just a few short years, from the rules that govern the
movement of physical objects to the hierarchical structure of natural
categories and even the relational structures among social and cultural
groups (Baillargeon 1994; Rogers and McClelland 2004; Legare and Harris
2016). Where does the information driving this rapid acquisition come
from? Undoubtedly, a sizeable portion comes from direct experience
observing and interacting with the world (Sloutsky and Fisher 2004;
Stahl and Feigenson 2015). But another important source of information
comes from the language people use to talk about the world (Landauer and
Dumais 1997; Rhodes, Leslie, and Tworek 2012). How similar is the
information from children's direct experience to the information
available in the language children hear?

Two lines of work suggest that they may be surprisingly similar. One
compelling area of work is the comparison of visual semantics learned by
congenitally blind people to those of their sighted peers. In several
domains that would seem at first blush to rely heavily on visual
information, such as verbs of visual perception (e.g., \emph{look},
\emph{see}), blind children and adults make semantic similarity
judgments that mirror their sighted peers (Landau, Gleitman, and Landau
2009; Bedny et al. 2019; Kim, Elli, and Bedny 2019).

A second line of evidence supporting the similarity of information in
perception and language is the broad success of statistical models
trained on language alone in approximating human judgments across a
variety of domains (Landauer and Dumais 1997; Mikolov et al. 2013;
Devlin et al. 2018; Brown et al. 2020). Even more compellingly, models
trained on both language and perceptual features for some words can
infer the perceptual features of linguistically related words entirely
from the covariation of language and perception (Johns and Jones 2012).

Still, there is reason to believe that some semantic features may be
harder to learn from language than these findings suggest. This is
because we rarely use language merely to provide running commentary on
the world around us; instead, we use language to talk about things that
diverge from our expectations or those of our conversational partner
(Grice 1975). People tend to avoid being over- or under-informative when
they speak. In particular, when referring to objects, people are
informative with respect to both the referential context and the typical
features of the referent (Westerbeek, Koolen, and Maes 2015;
Rubio-Fernández 2016). People tend to refer to an object that is typical
of its category with a bare noun (e.g., calling an orange carrot ``a
carrot''), but often specify when an object has an atypical feature
(e.g, ``a purple carrot''). Given these communicative pressures,
naturalistic language statistics may provide surprisingly little
evidence about what is typical (Willits, Sussman, and Amato 2008).

If parents speak to children in this minimally informative way, children
may be faced with input that emphasizes atypicality in relation to world
knowledge they do not yet have. For things like carrots---which children
learn about both from perception and from language---this issue may be
resolved by integrating both sources of information. Likely almost all
of the carrots children see are orange, and hearing an atypical exemplar
noted as a ``purple carrot'' may make little difference in their
inferences about the category of carrots more broadly. But for things to
which they lack perceptual access---such as rare objects, unfamiliar
social groups, or inaccessible features like the roundness of the
Earth---much of what they learn must come from language (Harris and
Koenig 2006). If language predominantly notes atypical features rather
than typical ones, children may overrepresent atypical features as they
learn the way things in the world tend to be.

On the other hand, parents may speak to children differently from the
way they speak to other adults. Parents' speech may reflect typical
features of the world more veridically, or even emphasize typical
features in order to teach children about the world. Parents alter their
speech to children along a number of structural dimensions, using
simpler syntax and more reduplications (Snow 1972). Their use of
description may reflect similar alignment to children's abilities by
emphasizing typical feature information children are still learning.

We examine the typicality of adjectives with respect to the nouns they
describe in a large, diverse corpus of parent-child interactions
recorded in children's homes to ask whether parents talking to their
children tend to use adjectives to mark atypical features. We find that
they do: Parents overwhelmingly choose to mention atypical rather than
typical features. We also find that parents use adjectives differently
over the course of children's development, noting highly typical
features more often to younger children. We additionally compare
parents' speech to a corpus of adult-adult speech and find that parents'
use of description when talking to children is quite similar to adults'
use of description when talking to other adults, and becomes more so as
children get older.

We then ask whether the co-occurrence structure of language nonetheless
captures typicality information by testing whether distributional
semantics models trained on child-directed speech and adult-directed
text capture adjective-noun typicality. We find that relatively little
typical feature information is represented in these semantic spaces. We
also test whether two more advanced language models, BERT and GPT-3,
capture typicality, and find that the latter does fairly well. These
models are unlikely to reflect children's learning mechanisms or
language input, but tell us what kinds of typicality information are
learnable from language in principle.

Children's \emph{own} speech offers a window into how children treat
adjectives: do children choose to remark on atypical features
themselves? We examine children's speech in the same corpus of
parent-child interactions and find that children too mostly remark on
the atypical rather than typical features of things. Though this
observational finding cannot provide definitive evidence that children
use description to be selectively informative about atypical features,
it suggests that even early in life their speech is shaped by adults'
pattern of selective description.

\hypertarget{adjective-typicality}{%
\section{Adjective typicality}\label{adjective-typicality}}

In order to determine whether parents use adjectives mostly to mark
atypical features of categories, we analyzed caregiver speech from a
large corpus of parent-child interactions, as well as adult-adult speech
as a comparison. We extracted adjectives and the nouns they modified
from caregiver speech, and asked a sample of Amazon Mechanical Turkers
to judge how typical the property described by each adjective was for
the noun it modified. We then examined both the broad features of this
typicality distribution and the way it changes over development.

\hypertarget{corpora}{%
\subsection{Corpora}\label{corpora}}

We used data from the Language Development Project, a large-scale,
longitudinal corpus of parent-child interactions recorded in children's
homes. Families were recruited to be representative of the Chicagoland
area in both socio-economic and racial composition; all families spoke
English at home (Goldin-Meadow et al. 2014). Recordings were taken in
the home every 4 months from when the child was 14 months old until they
were 58 months old, resulting in 12 timepoints. Each recording was of a
90-minute session in which parents and children were free to behave and
interact as they liked.

Our sample consisted of 64 typically-developing children and their
caregivers with data from at least 4 timepoints (\emph{mean} = 11.3
timepoints). Together, this resulted in a total of 641,402 parent
utterances and 368,348 child utterances.

As an adult-adult speech comparison, we used data from the Conversation
Analytic British National Corpus, a corpus of naturalistic, informal
conversations in people's everyday lives (Albert, Ruiter, and Ruiter
2015; Coleman et al. 2012). We excluded any conversations with child
participants, for a total of 99,305 adult-adult utterances.

\hypertarget{stimulus-selection}{%
\subsection{Stimulus Selection}\label{stimulus-selection}}

We parsed each utterance in our corpora using UDPipe, an automated
dependency parser, and extracted adjectives and the nouns they modified.
This set contained a number abstract or evaluative adjective-noun pairs
whose typicality would be difficult to classify (e.g.,
``good''--``job''; ``little''--``bit''). To resolve this issue, we used
human judgments of words' concreteness to identify and exclude
non-concrete adjectives and nouns (Brysbaert, Warriner, and Kuperman
2014). We retained for analysis only pairs in which both the adjective
and noun were in the top 25\% of concreteness ratings (e.g., ``dirty''
-- ``dish''; ``green'' -- ``fish''). Additionally, one common adjective
that is used abstractly and evaluatively in British English but is
concrete in American English (\emph{bloody}) was excluded from the set
of pairs from the CABNC.

\begin{table}[tb]
\centering
\begin{tabular}{llrrrr}
  \hline
utterance & pair & rating 1 & rating 2 & rating 3 & mean \\ 
  \hline
especially with wooden shoes. & wooden-shoe &   2 &   2 &   2 & 2.00 \\ 
  you like red onions? & red-onion &   5 &   3 &   4 & 3.60 \\ 
  the garbage is dirty. & dirty-garbage &   7 &   6 &   6 & 6.00 \\ 
   \hline
\end{tabular}
\caption{Sample typicality ratings from three human coders for three adjective-noun pairs drawn from the corpus. Note that means may be slightly different from the mean of the three ratings shown here because some pairs have more than three ratings.} 
\label{tab:utt_table}
\end{table}

Our final sample included 6,370 unique adjective-noun pairs drawn from
7,471 parent utterances, 2,775 child utterances, and 1,867 adult-adult
utterances. The pairs were combinations of 1,498 distinct concrete nouns
and 1,388 distinct concrete adjectives. We compiled these pairs and
collected human judgments on Amazon Mechanical Turk for each pair, as
described below. Table \ref{tab:utt_table} contains example utterances
from the final set and typicality judgments from our human raters.

\hypertarget{participants}{%
\subsection{Participants}\label{participants}}

Each participant rated 35 adjective-noun pairs, and we aimed for each
pair to be rated five times, for a total of 910 rating tasks.
Participants were allowed to rate more than one set of pairs and were
paid \$0.80 per task. Distribution of pairs was balanced using a MongoDB
database that tracked how often sets of pairs had been rated. If a
participant allowed their task to expire with the task partially
complete, we included those ratings and re-recruited the task. Overall,
participants completed 32,461 ratings. After exclusions using an
attention check that asked participants to simply choose a specific
number on the scale, we retained 32,293 judgments, with each
adjective--noun pair retaining at least two judgments.

\hypertarget{design-and-procedure}{%
\subsection{Design and Procedure}\label{design-and-procedure}}

To evaluate the typicality of the adjective--noun pairs that appeared in
parents' speech, we asked participants on Amazon Mechanical Turk to rate
each pair. Participants were presented with a question of the form ``How
common is it for a cow to be a brown cow?'' and asked to provide a
rating on a seven-point scale: (1) never, (2) rarely, (3) sometimes, (4)
about half the time, (5) often, (6) almost always, (7) always. We also
gave participants the option to select ``Doesn't make sense'' if they
could not understand what the adjective-noun pair would mean. Pairs that
were marked with ``Doesn't make sense'' by two or more participants were
excluded from the final set of pairs: 1,591 pairs were excluded at this
stage, for a final set of 4,779 rated adjective-noun pairs.

\hypertarget{results}{%
\subsection{Results}\label{results}}

We combined the human typicality ratings with usage data from our
corpora to examine the extent to which parents, children, and adults
speaking to other adults use language to describe typical and atypical
features. In our analyses, we token-weighted these judgments, giving
higher weight to pairs that occurred more frequently in speech. However,
results are qualitatively identical and all significant effects remain
significant when examined on a type level.

If caregivers speak informatively to convey what is atypical or
surprising in relation to their own sophisticated world knowledge, we
should see that caregiver description is dominated by adjectives that
are sometimes or rarely true of the noun they modify. If instead
child-directed speech privileges redundant information, perhaps to align
to young children's limited world knowledge, caregiver description
should yield a distinct distribution dominated by highly typical
modifiers. As we predicted, we found that parents' description
predominantly focuses on features that are atypical (Figure
\ref{fig:distribution-plot}).

\begin{figure}[tb]

{\centering \includegraphics{figs/distribution-plot-1} 

}

\caption{Density plots showing parents' use of atypical and typical adjective-noun pairs across their child's age.}\label{fig:distribution-plot}
\end{figure}

To confirm this effect statistically, we centered the ratings
(i.e.~``about half'' was coded as 0), and then predicted the rating on
each trial with a mixed effect model with only an intercept and a random
effect of noun (\texttt{typicality $\sim$ 1 + (1|noun)}). The intercept
was reliably negative, indicating that adjectives tend to refer to
atypical features of objects (\(\beta =\) -0.85, \(t =\) -28.611, \(p\)
\textless{} .001). We then re-estimated these models separately for each
age in the corpus, and found a reliably negative intercept for every age
group (smallest effect \(\beta_{14} =\) -0.684, \(t =\) -9.063, \(p\)
\textless{} .001). Even when talking with very young children, caregiver
speech is structured according to the kind of communicative pressures
observed in adult-adult conversation in the lab.

To examine whether this holds for naturalistic adult-adult conversation,
we performed the same analyses on usage of adjective-noun pairs in
adult-adult speech in the Conversation Analytic British National Corpus.
The overall distribution of adjective-noun typicality is remarkably
similar between child-directed and adult-directed speech (Figure
\ref{fig:cabnc-parent-overall}). Fitting the same mixed-effects model to
the adult-directed data, we found that the intercept was reliably
negative, indicating that adult-adult speech also predominantly
highlights atypical features (\(\beta =\) -0.925, \(t =\) -29.761, \(p\)
\textless{} .001).

\begin{figure}[tb]

{\centering \includegraphics{figs/cabnc-parent-overall-1} 

}

\caption{Density plots showing use of atypical and typical adjective-noun pairs by parents speaking to children and adults speaking to other adults.}\label{fig:cabnc-parent-overall}
\end{figure}

Returning to caregiver speech, while descriptions at every age tended to
point out atypical features (as in adult-adult speech), this effect
changed in strength over development. As predicted, an age effect added
to the previous model was reliably negative, indicating that parents of
older children are relatively more likely to focus on atypical features
(\(\beta =\) -0.071, \(t =\) -3.852, \(p\) \textless{} .001). In line
with the idea that caregivers adapt their speech to their children's
knowledge, it seems that caregivers are more likely to provide
description of typical features for their young children, compared with
older children. As a second test of this idea, we defined adjectives as
highly typical if Turkers judged them to be `often', `almost always', or
`always' true. We predicted whether each judgment was highly typical
from a mixed-effects logistic regression with a fixed effect of age
(log-scaled) and a random effect of noun. Age was a highly reliable
predictor (\(\beta =\) -0.688, \(t =\) -3.78, \(p\) \textless{} .001).
While children at all ages hear more talk about what is atypically true
(Figure \ref{fig:distribution-plot}), younger children hear relatively
more talk about what is typically true than older children do (Figure
\ref{fig:prototypical-plot}).

\begin{figure}[!tb]

{\centering \includegraphics{figs/prototypical-plot-1} 

}

\caption{Proportion of caregiver description that is about highly typical features (often, almost always, or always true), as a function of age.}\label{fig:prototypical-plot}
\end{figure}

What about children themselves---do they tend to remark on the atypical
rather than the typical features of things? We analyzed children's own
use of description and found that, following the pattern of parent
speech and adult-adult speech, they predominantly mention atypical
rather than typical features (Figure \ref{fig:child-ridge-plot}). The
fact that children are remarking on atypical features is intriguing, but
it would be premature to conclude that they are doing so to be
selectively informative. Note also that especially at young ages,
children produce few adjective-noun pairs---they are not producing any
at 14 months old, our earliest timepoint---so our data on children's
speech is somewhat sparse. We discuss potential interpretations of this
finding further in the Conclusion.

\begin{figure}[tb]

{\centering \includegraphics{figs/child-ridge-plot-1} 

}

\caption{Density plots showing children's use of atypical and typical adjective-noun pairs across age.}\label{fig:child-ridge-plot}
\end{figure}

\hypertarget{discussion}{%
\subsection{Discussion}\label{discussion}}

In sum, we find robust evidence that language is used to discuss
atypical, rather than typical, features of the world. Description in
caregiver speech seems to largely mirror the usage patterns that we
observed in adult-to-adult speech, suggesting that these patterns arise
from general communicative pressures. Interestingly, the descriptions
children hear change over development, becoming increasingly focused on
atypical features. The higher prevalence of typical descriptors in early
development may help young learners learn what is typical; however, even
at the earliest point we measured, the bulk of language input describes
atypical features.

This usage pattern aligns with the idea that language is used
informatively in relation to background knowledge about the world. It
may pose a problem, however, for young language learners with
still-developing world knowledge. If language does not transparently
convey the typical features of objects, and instead (perhaps
misleadingly) notes the atypical ones, how might children come to learn
what objects are typically like? One possibility is that information
about typical features is captured in more complex regularities across
many utterances. If this is true, language may still be an important
source of information about typicality as children may be able to
extract more accurate typicality information by tracking second-order
co-occurrence.

\hypertarget{extracting-typicality-from-language-structure}{%
\section{Extracting Typicality from Language
Structure}\label{extracting-typicality-from-language-structure}}

Much information can be gleaned from language that does not seem
available at first glance. From language alone, simple distributional
learning models can recover enough information to perform comparably to
non-native college applicants on the Test of English as a Foreign
Language (Landauer and Dumais 1997). Recently, Lewis, Zettersten, and
Lupyan (2019) demonstrated that even nuanced feature information may be
learnable through distributional semantics alone, without any complex
inferential machinery. Further, experiments with adults and children
suggest that co-occurrence regularities may help structure semantic
knowledge (Unger, Savic, and Sloutsky 2020; Savic, Unger, and Sloutsky
2023, 2022). Here, we ask whether a simple distributional semantics
model trained on the language children hear can capture typical feature
information. Further, we test whether a distributional semantics model
trained on a larger corpus of adult-directed text as well as two more
sophisticated language models capture adjective-noun typicality. These
models are trained on more and different language than is available to
children, but tell us more about whether and how typicality information
is learnable by applying simple learning objectives to text.

\hypertarget{method}{%
\subsection{Method}\label{method}}

To test this possibility, we trained word2vec---a distributional
semantics model---on the same corpus of child-directed speech used in
our first set of analyses. Word2vec is a neural network model that
learns to predict words from the contexts in which they appear. This
leads word2vec to encode words that appear in similar contexts as
similar to one another (Firth 1957).

We used the continuous-bag-of-words (CBOW) implementation of word2vec in
the \texttt{gensim} package (Řehůřek and Sojka 2010). We trained the
model using a surrounding context of 5 words on either side of the
target word and 100 dimensions (weights in the hidden layer) to
represent each word. After training, we extracted the hidden layer
representation of each word in the model's vocabulary---these are the
vectors used to represent these words.

If the model captures information about the typical features of objects,
we should see that the model's noun-adjective word pair similarities are
correlated with the typicality ratings we elicited from human raters.
For a second comparison, we also used an off-the-shelf implementation of
word2vec trained on Wikipedia (Mikolov et al. 2018). While the Language
Development Project corpus likely underestimates the amount of structure
in children's linguistic input, Wikipedia likely overestimates it.

While word2vec straightforwardly represents what can be learned about
word similarity by associating words with similar contexts, it does not
represent the cutting edge of language modeling. Perhaps more
sophisticated models trained on larger corpora would represent these
typicalities better. To test this, we asked how BERT (Devlin et al.
2018) and GPT-3 (Brown et al. 2020) represent typicality. BERT is a
masked language model trained on BookCorpus and English Wikipedia, which
represents the probability of words occurring in slots in a phrase. We
gave BERT phrases of the form ``\_\_\_\_ apple'', and asked it the
probability of different adjectives filling the empty slot.

GPT-3 is a generative language model trained on large quantities of
internet text, including Wikipedia, book corpora, and web page text from
crawling the internet. Because it is a generative language model, we can
ask GPT-3 the same question we asked human participants directly and it
can generate a text response. We prompted the \texttt{davinci-text-003}
instance of GPT-3 questions of the form: ``You are doing a task in which
you rate how common it is for certain things to have certain features.
You respond out of the following options: Never, Rarely, Sometimes,
About half the time, Often, Almost always, or Always. How common is it
for a cow to be a brown cow?'' Because BERT and GPT-3 are trained on
more and different kinds of language than what children hear, results
from these models likely do not straightforwardly represent the
information available to children in language. However, results from
BERT and GPT-3 can indicate the challenges language models face in
representing world knowledge when the language people use emphasizes
remarkable rather than typical features.

\hypertarget{results-1}{%
\subsection{Results}\label{results-1}}

We find that similarities in the model trained on the Language
Development Project corpus have near zero correlation with human
adjective--noun typicality ratings (\(r =\) 0.05, \(p =\) .001).
However, our model does capture other meaningful information about the
structure of language, such as similarity within part of speech
categories. Comparing with pre-existing large-scale human similarity
judgements for word pairs, our model shows significant correlations
(correlation with wordsim353 similarities of noun pairs, 0.28;
correlation with simlex similarities of noun, adjective, and verb pairs,
0.16). This suggests that statistical patterns in child-directed speech
are likely insufficient to encode information about the typical features
of objects, despite encoding at least some information about word
meaning more broadly.

However, the corpus on which we trained this model was small; perhaps
our model did not get enough language to draw out the patterns that
would reflect the typical features of objects. To test this possibility,
we asked whether word vectors trained on a much larger corpus---English
Wikipedia---correlate with typicality ratings. This model's similarities
were significantly correlated with human judgments, although the
strength of the correlation was still fairly weak (\(r =\) 0.338, \(p\)
\textless{} .001). How do larger and more sophisticated language models
fare? Like Wikipedia-trained word2vec, BERT's probabilities were
significantly correlated with human judgments, though weakly so (\(r =\)
0.154, \(p\) \textless{} .001). However, GPT-3's ratings were much
better aligned with human judgments (\(r =\) 0.574, \(p\) \textless{}
.001).

Similarity judgments produced by our models reflect many dimensions of
similarity, but our human judgments reflect only typicality. To account
for this fact and control for semantic differences among the nouns in
our set, we performed a second analysis in which we considered only the
subset of 109 nouns that had both a high-typicality (rated as at least
``often'') and a low-typicality (rated as at most ``sometimes'')
adjective. We then asked whether the word2vec models rated the
high-typicality adjective as more similar to the noun it modified than
the low-typicality adjective. The LDP model correctly classified 49 out
of 109 (0.45), which was not different from chance (\(p =\) .338). The
Wikipedia-trained word2vec model correctly classified 84 out of 109
(0.771), which was better than chance according to a binomial test,
though not highly accurate (\(p\) \textless{} .001). Figure
\ref{fig:halfs} shows the word2vec models' similarities for the 109
nouns and their typical and atypical adjectives alongside scaled average
human ratings.

The analogous analysis on BERT asks whether the model rates the
high-typicality adjective as more likely to come before the noun than
the low typicality adjective (e.g., P(``red'') \textgreater{}
P(``brown'') in ``\_\_\_\_ apple''). BERT correctly classified 66 out of
109 (0.606), which is significantly better than chance (\(p =\) .035).
However, BERT's performance was directionally less accurate than
Wikipedia-trained word2vec: though BERT is a more sophisticated model,
it does not capture adjective-noun typicality better than word2vec in
this analysis. GPT-3 performs much better than BERT and the word2vec
models, with 96 out of 109 (0.881; \(p\) \textless{} .001). Figure
\ref{fig:bert-gpt} shows BERT and GPT-3 ratings for the 109 nouns and
their typical and atypical adjectives alongside scaled average human
ratings.

\begin{figure}[!tb]

{\centering \includegraphics{figs/halfs-1} 

}

\caption{Plots of word2vec noun-adjective similarities for nouns for which there was at least one atypical adjective (rated at most "sometimes"), and at least one typical adjective (rated at least "often").}\label{fig:halfs}
\end{figure}

\begin{figure}[!tb]

{\centering \includegraphics{figs/bert-gpt-1} 

}

\caption{Plots of BERT and GPT-3 noun-adjective similarities for nouns for which there was at least one atypical adjective (rated at most "sometimes"), and at least one typical adjective (rated at least "often").}\label{fig:bert-gpt}
\end{figure}

\hypertarget{general-discussion}{%
\section{General Discussion}\label{general-discussion}}

Language provides children a rich source of information about the world.
However, this information is not always transparently available: because
language is used to comment on the atypical, it does not perfectly
mirror the world. Among adult conversational partners whose world
knowledge is well-aligned, this principle allows people to converse
informatively and avoid redundancy. But between a child and caregiver
whose world knowledge is asymmetric, this pressure competes with other
demands: what is minimally informative to an adult may be misleading to
a child. Our results show that this pressure structures language to
create a peculiar learning environment, one in which caregivers
predominantly point out the atypical features of things.

How, then, do children learn about the typical features of things? While
younger children may gain an important foothold from hearing more
description of typical features, they still face language dominated by
atypical description. When we looked at more nuanced ways of extracting
information from language (which may or may not be available to the
developing learner), we found that two word2vec models, one trained on
child-directed language and one trained on adult-adult language, did not
capture typicality very well. Even BERT, a language model trained on
much more text and with a more complex architecture, did not perform
better than a Wikipedia-trained word2vec model in reflecting typicality.
This may be because these models are designed to capture language
statistics, with BERT in particular capturing which words are likely to
occur following one another---and as we show in our corpus analyses,
adjective-noun pairs that come together often reflect atypicality rather
than typicality. Note that a consistent \emph{inverse}
relationship---rating high-typicality pairs as \emph{less} similar or
\emph{less} probable---would also be evidence that these models capture
typicality, but the word2vec models and BERT do not evince this pattern
either. However, GPT-3 captured typicality quite well, suggesting that
the way people structure language to emphasize atypicality is not
necessarily an impediment for much larger models' representation of
typicality. Further work remains to understand how GPT-3 comes to
represent typicality relationships so much better than the smaller
models we tested. Overall, a large language model trained on text much
greater in quantity and different in quality from child-directed
language did capture adjective-noun typicality well, but models with
simpler learning mechanisms and language input more similar to what is
available to children did not.

Of course, perceptual information from the world may simplify the
problem of learning about typicality. In many cases, perceptual
information may swamp information from language; children likely see
enough orange carrots in the world to outweigh hearing ``purple
carrot.'' It remains unclear, however, how children learn about
categories for which they have scarcer evidence. Indeed, language
information likely swamps perceptual information for many other
categories, such as abstract concepts or those that cannot be learned
about by direct experience. If such concepts pattern similarly to the
concrete objects analyzed here, children are in a particularly difficult
bind.

It is also possible that other cues from language and interaction
provide young learners with clues to what is typical or atypical, and
these cues are uncaptured by our measure of usage statistics. Caregivers
may highlight when a feature is typical by using certain syntactic
constructions, such as generics (e.g., ``tomatoes are red''). Caregivers
may also mark the atypicality of a feature using extralinguistic cues,
e.g., by demonstrating surprise using prosody and facial expressions.
Such cues from language and interaction may provide key information in
some cases; however, given the sheer frequency of atypical descriptors,
it seems unlikely that they are consistently well-marked.

Another possibility is that children expect language to be used
informatively at a young age. Under this hypothesis, their language
environment is not misleading at all, even without additional cues from
caregivers. Children as young as two years old tend to use words to
comment on what is new rather than what is known or assumed (Baker and
Greenfield 1988). Children may therefore expect adjectives to comment on
surprising features of objects. If young children expect adjectives to
mark atypical features (Horowitz and Frank 2016), they can use
description and the lack thereof to learn more about the world. Our
finding that children themselves mostly remark on atypical rather than
typical features of things is consistent with this possibility, though
does not provide strong evidence that children understand to use
description informatively. We will further investigate this question by
studying children's interpretation of adjectives in Chapter 3.

Across our analyses, language is used with remarkable consistency:
people talk about the atypical. Though parents might reasonably be
broadly over-informative in order to teach their children about the
world, this is not the case. This presents a potential puzzle for young
learners who have limited world knowledge and limited pragmatic
inferential abilities. Perceptual information and nascent pragmatic
abilities may help fill in the gaps, but much remains to be explored to
link these explanations to actual learning. Communication pressures are
pervasive forces structuring the language children hear, and further
work can disentangle whether children capitalize on them or are misled
by them in learning about the world.

\hypertarget{how-adults-use-contrastive-inference-to-learn-about-new-categories}{%
\chapter{How adults use contrastive inference to learn about new
categories}\label{how-adults-use-contrastive-inference-to-learn-about-new-categories}}

When referring to a \emph{big red dog} or a \emph{hot-air balloon}, we
often take care to describe them---even when there are no other dogs or
balloons around. Speakers use more description when referring to objects
with atypical features (e.g., a yellow tomato) than typical ones (e.g.,
a red tomato; see Chapter 1 and Bergey, Morris, and Yurovsky 2020;
Mitchell, Reiter, and Deemter 2013; Westerbeek, Koolen, and Maes 2015;
Rubio-Fernández 2016). This selective marking of atypical objects
potentially supplies useful information to listeners: they have the
opportunity to not only learn about the object at hand, but also about
its broader category.

Horowitz and Frank (2016) demonstrated that, combined with other
contrastive cues (e.g., ``Wow, this one is a zib. This one is a TALL
zib''), prenominal adjectives prompted adults and children to infer that
the described referent was less typical than one that differed on the
mentioned feature (e.g., a shorter zib). This work provided a useful
demonstration that adjective use can contribute to inferences about
feature typicality, though it did not isolate the effect of adjectives
specifically. Their experiments used several contrastive cues, such as
prosody (contrastive stress on the adjective: ``TALL zib''),
demonstrative phrases that may have marked the object as unique (``this
one'') and expressions of surprise at the object (``wow''), and
participants may have inferred the object was atypical primarily from
these cues and not from the adjective. In Chapter 2, we test whether
adjective use alone prompts an inference of atypicality with respect to
the category's feature distribution: when you hear ``purple toma,'' do
you infer that \emph{fewer} tomas in general are purple?

If listeners do make contrastive inferences about typicality, it may not
be as simple as judging that a described referent is atypical.
Description can serve many purposes. If a descriptor is needed to
distinguish between two present objects, it may not have been used to
mark atypicality. For instance, in the context of a bin of heirloom
tomatoes, a speaker who wants a red one in particular might specify that
they want a ``red tomato'' rather than just asking for a ``tomato.'' In
this case, the adjective ``red'' is being used contrastively with
respect to reference, and not to mark atypicality. If reference is the
primary motivator of speakers' word choice, as implicitly assumed in
much research (e.g., Pechmann 1989; Engelhardt, Barış Demiral, and
Ferreira 2011; Arts et al. 2011), then people should draw no further
inferences once the need for referential disambiguation explains away a
descriptor like ``red.'' On this reference-first view, establishing
reference has priority in understanding the utterance, and any further
inferences are blocked if the utterance is minimally informative with
respect to reference. If, on the other hand, pragmatic reasoning weighs
multiple goals simultaneously--here, reference and conveying
typicality--people may integrate typicality as just one factor the
speaker considers in using description, leading to graded inferences
about the referent's identity and about its category's features.

In two experiments, we used an artificial language task to set up just
this kind of learning situation. We manipulated the contexts in which
listeners hear adjectives modifying novel names of novel referents.
These contexts varied in how useful the adjective was to identify the
referent: some contexts the adjectives were necessary for reference, and
in others they were unhelpful. On a \emph{reference-first view}, use of
an adjective that was necessary for reference can be explained away and
should not prompt further inferences about typicality---an atypicality
inference would be blocked. If, on the other hand, people take into
account speakers' multiple reasons for using adjectives without giving
priority to reference (the \emph{probabilistic weighing view}), they may
alter their inferences about typicality across these contexts in a
graded way: if an adjective was necessary for reference, it may prompt
slightly weaker inferences of atypicality; if an adjective was redundant
with respect to reference, it may be inferred to mark atypicality more
strongly. Further, these contexts may also prompt distinct inferences
when no adjective is used: for instance, when an adjective is necessary
to identify the referent but elided, people may infer that the elided
feature is particularly typical. To account for the multiple ways
context effects might emerge, we analyze both of these possibilities.
Overall, we asked whether listeners infer that these adjectives identify
atypical features of the named objects, and whether the strength of this
inference depends on the referential ambiguity of the context in which
adjectives are used.

\begin{figure}[!tb]

{\centering \includegraphics{figs/e2-aliens-1} 

}

\caption{Experiment 1 stimuli. In the above example, the critical feature is size and the object context is a within-category contrast: the alien on the right has two same-shaped objects that differ in size.}\label{fig:e2-aliens}
\end{figure}

\hypertarget{experiment-1}{%
\section{Experiment 1}\label{experiment-1}}

\hypertarget{method-1}{%
\subsection{Method}\label{method-1}}

\hypertarget{participants.}{%
\subsubsection{Participants.}\label{participants.}}

240 participants were recruited from Amazon Mechanical Turk. Half of the
participants were assigned to a condition in which the critical feature
was color (red, blue, purple, or green), and the other half of
participants were assigned to a condition in which the critical feature
was size (small or big). Participants were paid \$0.30. Participants
were told the task was estimated to take 3 minutes and on average took
118 seconds to complete the task (not including reading the consent
form).

\hypertarget{stimuli-procedure.}{%
\subsubsection{Stimuli \& Procedure.}\label{stimuli-procedure.}}

Stimulus displays showed two alien interlocutors, one on the left side
(Alien A) and one on the right side (Alien B) of the screen, each with
two novel fruit objects beneath them (Figure \ref{fig:e2-aliens}). Alien
A, in a speech bubble, asked Alien B for one of its fruits (e.g., ``Hey,
pass me the big toma''). Alien B replied, ``Here you go!'' and the
referent disappeared from Alien B's side and reappeared on Alien A's
side. Note that the participants do not make a referent choice in this
experiment; the measure of interest is their typicality judgments of the
objects' features, described below.

We manipulated three factors: utterance type, feature type, and context
type. We prioritized utterance type as a within-subjects manipulation
because it was the central manipulation of interest. We also prioritized
context type because another central question was whether context would
alter the effect of utterance. We manipulated the critical feature type
(color or size) between subjects, to maximize our use of the set of
novel stimuli without showing any participant the same novel shape on
more than one trial.

Utterance type and context type were fully crossed within subjects.
Utterance type had two levels: \emph{adjective noun} (e.g., ``Hey, pass
me the big toma'' or ``Hey, pass me the blue toma'') or \emph{noun}
(e.g., ``Hey, pass me the toma''). Context type had three levels:
within-category contrast, between-category contrast, and same feature
(Figure \ref{fig:e2-wppl-plot}). In the within-category contrast
condition, Alien B possessed the target object and another object of the
same shape, but with a different feature value (e.g., a big toma and a
small toma). In the between-category contrast condition, Alien B
possessed the target object and another object of a different shape, and
with a different feature value (e.g., a big toma and a small blicket).
In the same feature condition, Alien B possessed the target object and
another object of a different shape and with the same feature as the
target (e.g., a big toma and a big dax). Thus, in the within-category
contrast condition, the descriptor was necessary to distinguish the
referent; in the between-category contrast condition it was unnecessary
but potentially helpful; and in the same feature condition it was
unnecessary and unhelpful.

Note that in all context conditions, the set of objects on screen was
the same in terms of the experiment design: there was a target (e.g.,
big toma), an object with the same shape as the target and a different
critical feature (e.g., small toma), an object with a different shape
from the target and the same critical feature (e.g., big dax), and an
object with a different shape from the target and a different critical
feature (e.g., small blicket). Context was manipulated by rearranging
these objects such that the relevant referents (the objects under Alien
B) differed and the remaining objects were under Alien A. Thus, in each
case, participants saw the target object and one other object that
shared the target object's shape but not its critical feature--they
observed the same kind of feature distribution of the target object's
category in each trial type.

The particular values of the features were chosen randomly for each
trial, and fruits were chosen randomly at each trial from 25 fruit
kinds. Ten of the 25 fruit drawings were adapted and redrawn from
Kanwisher et al. (1997); we designed the remaining 15 fruit kinds. Each
fruit kind had an instance in each of four colors (red, blue, green, or
purple) and two sizes (big or small).

Participants completed six trials. After each exchange between the alien
interlocutors, they made a judgment about the prevalence of the target's
critical feature in the target object's category. For instance, after
seeing a red blicket being exchanged, participants would be asked, ``On
this planet, what percentage of blickets do you think are red?'' They
answered on a sliding scale between zero and 100. In the size condition,
participants were asked, ``On this planet, what percentage of blickets
do you think are the size shown below?'' with an image of the target
object they just saw available on the screen.

After completing the study, participants were asked to select which of a
set of alien words they had seen previously during the study. Four were
words they had seen, and four were novel lure words. Participants were
dropped from further analysis if they did not respond to at least 6 of
these 8 correctly (above chance performance as indicated by a one-tailed
binomial test at the \(p = .05\) level). This resulted in excluding 47
participants, leaving 193 for further analysis.

\hypertarget{results-2}{%
\subsection{Results}\label{results-2}}

Our key test is whether participants infer that a mentioned feature is
less typical than one that is not mentioned. In addition, we tested
whether inferences of atypicality are modulated by context. One way to
test this is to analyze the interaction between utterance type and
context, seeing if the difference between \emph{noun} and
\emph{adjective noun} utterances is larger when the adjective was highly
redundant or smaller when the adjective was necessary for reference.

We analyzed participants' judgments of the prevalence of the target
object's critical feature in its category. We began by fitting a maximum
mixed-effects linear model with effects of utterance type (\emph{noun}
or \emph{adjective noun}), context type (within category, between
category, or same feature, with between category as the reference
level), and critical feature (color or size) as well as all interactions
and random slopes of utterance type and context type nested within
subject. Random effects were removed until the model converged. The
final model included the effects of utterance type, context type, and
critical feature and their interactions, and a random slope of utterance
type by subject.

This model revealed a significant effect of utterance type
(\(\beta_{adjective} =\) -10.22, \(t =\) -3.374, \(p =\) .001), such
that prevalence judgments were lower when an adjective was used than
when it was not. Participants' inferences did not significantly differ
between color and size adjective conditions (\(\beta_{size} =\) 4.728,
\(t =\) 1.455, \(p =\) .146). Participants' inferences did not
significantly vary by context type (\(\beta_{within} =\) 3.924, \(t =\)
1.628, \(p =\) .104; \(\beta_{same} =\) -1.485, \(t =\) -0.618, \(p =\)
.537). There was not a significant interaction between context and
presence of an adjective in the utterance
(\(\beta_{within*adjective} =\) -1.578, \(t =\) -0.463, \(p =\) .644;
\(\beta_{same*adjective} =\) 2.131, \(t =\) 0.625, \(p =\) .532). That
is, participants did not significantly adjust their inferences based on
object context, nor did they make differential inferences based on the
combination of context and adjective use. However, they robustly
inferred that mentioned features were less prevalent in the target's
category than unmentioned features.

This lack of a context effect may be because people do not take context
into account, or because they make distinct inferences when an adjective
is \emph{not} used: for instance, when an adjective is necessary for
reference but elided, people may infer that the unmentioned feature is
very typical. This inference would lead to a difference between the
\emph{noun} and \emph{adjective noun} utterances in the within-category
context, but not because people are failing to attribute the adjective
to reference. To account for this possibility, we separately tested
whether there are effects of context among just the \emph{noun} trials
and just the \emph{adjective noun} trials. In each case, we fit a model
with effects of context type and critical feature as well as their
interaction and random slopes by subject. Participants did not
significantly adjust their inferences by context among only the
\emph{noun} trials (\(\beta_{within} =\) 3.945, \(t =\) 1.469, \(p =\)
.143; \(\beta_{same} =\) -1.456, \(t =\) -0.544, \(p =\) .587), though
numerically they made higher prevalence judgments in the within-category
context. That is, we did not find evidence here that people were
inferring a feature to be highly typical because it went unmentioned
when it was necessary for reference. Participants also did not
significantly adjust their inferences by context among only the
\emph{adjective noun} trials (\(\beta_{within} =\) 2.434, \(t =\) 1.159,
\(p =\) .247; \(\beta_{same} =\) 0.67, \(t =\) 0.319, \(p =\) .750),
though their judgments were numerically higher in the within-category
context. That is, we did not find evidence that people modulated their
typicality inferences based on the referential context among trials
where this inference could not have been driven by omission either.
Overall, we did not find evidence that participants significantly
adjusted their inferences based on context.

\hypertarget{discussion-1}{%
\subsection{Discussion}\label{discussion-1}}

Description is often used not to distinguish among present objects, but
to pick out an object's feature as atypical of its category. In
Experiment 1, we asked whether people would infer that a described
feature is atypical of a novel category after hearing it mentioned in an
exchange. We found that people robustly inferred that a mentioned
feature was atypical of its category, across both size and color
description. Further, participants did not use object context to
substantially explain away description. That is, even when description
was necessary to distinguish among present objects (e.g., there were two
same-shaped objects that differed only in the mentioned feature),
participants still inferred that the feature was atypical of its
category. This suggests that, in the case of hearing someone ask for a
``red tomato'' from a bin of many-colored heirloom tomatoes, a
tomato-naive person would infer that tomatoes are relatively unlikely to
be red.

Another interpretation of people's inferences in the size condition is
that they are due to size adjectives being relative gradable adjectives.
That is, the phrases ``big toma'' and ``small toma'' may inherently
carry the meaning ``big for a toma'' and ``small for a toma'' (which can
be interpreted as an aspect of the adjective's semantics, not
pragmatics; Kennedy 2007; Xiang et al. 2022; Tessler et al. 2020). It is
possible to attribute people's atypicality inferences in the size
condition to the relative gradable nature of size adjectives. However,
people also made these inferences about color adjectives, which are not
relative gradable adjectives. A purely semantic account also might
predict that people's inferences about color and size would be
different---for instance, that people would make larger atypicality
inferences about size than color---which we do not find. Though the
semantics of size adjectives may contribute to people's inferences of
atypicality in the size condition, we find it parsimonious here to
explain the color and size inferences by the same mechanism---pragmatic
reasoning.

\hypertarget{model}{%
\subsection{Model}\label{model}}

To formalize the inference that participants were asked to make, we
developed a model in the Rational Speech Act Framework (RSA, Frank and
Goodman 2012). In this framework, pragmatic listeners (\(L\)) are
modeled as drawing inferences about speakers' (\(S\)) communicative
intentions in talking to a hypothetical literal listener (\(L_{0}\)).
This literal listener makes no pragmatic inferences at all, evaluating
the literal truth of a statement (e.g., it is true that a red toma can
be called ``toma'' and ``red toma'' but not ``blue toma''), and chooses
randomly among all referents consistent with that statement. In planning
their referring expressions, speakers choose utterances that are
successful at accomplishing two goals: (1) making the listener as likely
as possible to select the correct object, and (2) minimizing their
communicative cost (i.e., producing as few words as possible). Note that
though determiners are not given in the model's utterances, the
assumption that the utterance refers to a specific reference is built
into the model structure, consistent with the definite determiners used
in the task. Pragmatic listeners use Bayes' rule to invert the speaker's
utility function, essentially inferring what the speaker's intention was
likely to be given the utterance they produced.

\[Literal: P_{Lit} = \delta\left(u,r\right)P\left(r\right)\]

\[Speaker: P_S\left(u \vert r\right) \propto \alpha \left(P_{Lit}\left(r \vert u\right) - C\right)\]

\[Listener: P_{Learn}\left(r \vert u\right) \propto P_s\left(u \vert r\right)P\left(r\right)\]

To allow the Rational Speech Act Framework to capture inferences about
typicality, we modified the Speaker's utility function to have an
additional term: the listener's expected processing difficulty. Speakers
may be motivated to help listeners to select the correct referent not
just eventually but as quickly as possible. People are both slower and
less accurate at identifying atypical members of a category as members
of that category (Rosch, Simpson, and Miller 1976; Dale, Kehoe, and
Spivey 2007). If speakers account for listeners' processing
difficulties, they should be unlikely to produce bare nouns to refer to
low typicality exemplars (e.g.~unlikely to call a purple carrot simply
``carrot''). This is roughly the kind of inference encoded in a
continuous semantics Rational Speech Act model (Degen et al. 2020).

We model the speaker as reasoning about the listener's label
verification process. Because the speed of verification scales with the
typicality of a referent, a natural way of modeling it is as a process
of searching for that particular referent in the set of all exemplars of
the named category, or alternatively of sampling that particular
referent from the set of all exemplars in that category,
\(P\left(r \vert Cat\right)\). On this account, speakers want to provide
a modifying adjective for atypical referents because the probability of
sampling them from their category is low, but the probability of
sampling them from the modified category is much higher (a
generalization of the size principle, Xu and Tenenbaum 2007). Typicality
is just one term in the speaker's utility, and thus is directly weighed
with the literal listener's judgment and against cost.

If speakers use this utility function, a listener who does not know the
feature distribution for a category can use a speaker's utterance to
infer it. Intuitively, a speaker should prefer not to modify nouns with
adjectives because they incur a cost for producing an extra word. If
they did use an adjective, it must be because they thought the learner
would have a difficult time finding the referent from a bare noun alone
because of typicality, competing referents, or both. To infer the true
prevalence of the target feature in the category, learners combine the
speaker's utterance with their prior beliefs about the feature
distribution.

We model the learner's prior about the prevalance of features in any
category as a \(\text{Beta}\) distribution with two parameters
\(\alpha\) and \(\beta\) that encode the number of hypothesized prior
psuedo-exemplars with the feature and without feature that the learner
has previously observed (e.g., one red dax and one blue dax). We assume
that the learner believes they have previously observed one hypothetical
psuedo-examplar of each type, which is a weak symmetric prior indicating
that the learner expects the target feature value to occur in half of
all members of a category on average, but would find many levels of
prevalence unsurprising. To model the learner's direct experience with
the category, we add the observed instances in the experiment to these
hypothesized prior instances. After observing one member of the category
with the target feature value and one without, the listener's prior is
thus updated to be \(\text{Beta}\left(2,\,2\right)\).

We used Bayesian data analysis to estimate the posterior mean
rationality parameter that participants are using to draw inferences
about speakers in both the color and size conditions. The absolute
values of these parameters are driven largely by the number of
pseudo-exemplars assumed by the listener prior to exposure; however,
differences between color and size within the model are interpretable.
We found that listeners inferred speakers to be directionally more
rational when using size adjectives (0.887 {[}0.626, 1.134{]}) than
color adjectives (0.604 {[}0.367, 0.833{]}), but the two inferred
confidence intervals were overlapping, suggesting that people treated
size and color adjectives similarly when making inferences about
typicality.

\begin{figure}[!tb]

{\centering \includegraphics{figs/e2-wppl-plot-1} 

}

\caption{Participants' prevalence judgments from Experiment 1, along with our model predictions. Participants consistently judged the target object as less typical of its category when the referent was described with an adjective (e.g., "Pass me the blue toma") than when it was not (e.g., "Pass me the toma"). This inference was not significantly modulated by object context (examples shown above each figure panel). Points indicate empirical means; error bars indicate 95\% confidence intervals computed by non-parametric bootstrapping. Solid horizontal lines indicate model predictions.}\label{fig:e2-wppl-plot}
\end{figure}

Figure \ref{fig:e2-wppl-plot} shows the predictions of our Rational
Speech Act model compared to empirical data from participants. The model
captures the trends in the data correctly, inferring that the critical
feature was less prevalent in the category when it was mentioned (e.g.,
``red dax'') than when it was not mentioned (e.g., ``dax''). The model
also infers the prevalence of the critical feature to be numerically
higher in the within-category condition, like people do. That is, in the
within-category condition when an adjective is used to distinguish
between referents, the model thinks that the target color is slightly
less atypical. When an adjective would be useful to distinguish between
two objects of the same shape but one is not used, the model infers that
the color of the target object is slightly more typical.

Overall, our model captures the inference people make: when the speaker
mentions a feature (e.g., ``the blue dax''), people infer that the
feature is less typical of the category (daxes are less likely to be
blue in general). It further captures that when the object context
requires an adjective for successful reference, people weaken this
atypicality inference only slightly, if at all. In contrast to a
reference-first view, which predicts that these two kinds of inferences
would trade off strongly--that is, using an adjective that is necessary
for reference would block the inference that it is marking
atypicality--the model captures the graded way in which people consider
these two communicative goals.

\hypertarget{experiment-2}{%
\section{Experiment 2}\label{experiment-2}}

In Experiment 1, we established that people can use contrastive
inferences to make inferences about the feature distribution of a novel
category. Additionally, we found that these two inferences do not seem
to trade off substantially: even if an adjective is necessary to
establish reference, people infer that it also marks atypicality. To
strengthen our findings in a way that would allow us to better detect
potential trade-offs between these two types of inference, in Experiment
2 we conducted a pre-registered replication of Experiment 1 with a
larger sample of participants. In addition, we tested how people's
prevalence judgments from utterances with and without an adjective
compare to their null inference about feature prevalence by adding a
control utterance condition: an alien utterance, which the participants
could not understand. This also tests the model assumption we made in
Experiment 1: that after seeing two exemplars of the target object with
two values of the feature (e.g., one green and one blue), people's
prevalence judgments would be around 50\%. In addition to validating
this model assumption, we more strongly tested the model here by
comparing predictions from same model, with parameters inferred from the
Experiment 1 data, to data from Experiment 2. Our pre-registration of
the method, recruitment plan, exclusion criteria, and analyses can be
found on the Open Science Framework: \url{https://osf.io/s8gre} .

\hypertarget{method-2}{%
\subsection{Method}\label{method-2}}

\hypertarget{participants.-1}{%
\subsubsection{Participants.}\label{participants.-1}}

A pre-registered sample of 400 participants was recruited from Amazon
Mechanical Turk. Half of the participants were assigned to a condition
in which the critical feature was color (red, blue, purple, or green),
and half of the participants were assigned to a condition in which the
critical feature was size (small or big).

\hypertarget{stimuli-procedure.-1}{%
\subsubsection{Stimuli \& Procedure.}\label{stimuli-procedure.-1}}

The stimuli and procedure were identical to those of Experiment 2, with
the following modifications. Two factors, utterance type and object
context, were fully crossed within subjects. Object context had two
levels: within-category contrast and between-category contrast. In the
within-category context condition, Alien B possessed the target object
and another object of the same shape, but with a different value of the
critical feature (color or size). In the between-category contrast
condition, Alien B possessed the target object and another object of a
different shape, and with a different value of the critical feature.
Thus, in the within-category contrast condition, an adjective is
necessary to distinguish the referent; in the between-category contrast
condition it is unnecessary but potentially helpful. There were three
utterance types: adjective, no adjective, and alien utterance. In the
two alien utterance trials, the aliens spoke using completely unfamiliar
utterances (e.g., ``Zem, noba bi yix blicket''). Participants were told
in the task instructions that sometimes the aliens would talk in a
completely alien language, and sometimes their language will be partly
translated into English. To keep participants from making inferences
about the content of the alien utterances using the utterance content of
other trials, both alien language trials were first; other than this
constraint, trial order was random. We manipulated the critical feature
type (color or size) between subjects.

After completing the study, participants were asked to select which of a
set of alien words they had seen previously during the study. Four were
words they had seen, and four were novel lure words. Participants were
dropped from further analysis if they did not meet our pre-registered
criteria of responding to at least 6 of these 8 correctly (above chance
performance as indicated by a one-tailed binomial test at the
\(p = .05\) level) and answering all four color perception check
questions correctly. Additionally, six participants were excluded
because their trial conditions were not balanced due to an error in the
run of the experiment. This resulted in excluding 203 participants,
leaving 197 for further analysis. In our pre-registration, we noted that
we anticipated high exclusion rates, estimating that approximately 150
people per condition would be sufficient to test our hypotheses.

\hypertarget{results-3}{%
\subsection{Results}\label{results-3}}

We began by fitting a pre-registered maximum mixed-effects linear model
with effects of utterance type (alien utterance, adjective, or no
adjective; alien utterance as reference level), context type (within
category or between category), and critical feature (color or size) as
well as all interactions and random slopes of utterance type and context
type nested within subject. Random effects were removed until the model
converged, which resulted in a model with all fixed effects, all
interactions and a random slope of utterance type by subject. The final
model revealed a significant effect of the no adjective utterance type
compared to the alien utterance type (\(\beta =\) 7.476, \(t =\) 2.798,
\(p =\) .005) and no significant effect of the adjective utterance type
compared to the alien utterance type (\(\beta =\) -0.641, \(t =\)
-0.244, \(p =\) .808). The effects of context type (within-category or
between-category) and adjective type (color or size) were not
significant (\(\beta_{within} =\) -2.699, \(t_{within} =\) -1.228,
\(p_{within} =\) .220; \(\beta_{size} =\) 4.435, \(t_{size} =\) 1.33,
\(p_{size} =\) .185). There were marginal interactions between the
adjective utterance type and the size condition (\(\beta =\) -6.561,
\(t =\) -1.724, \(p =\) .086), the adjective utterance type and the
within-category context (\(\beta =\) 5.767, \(t =\) 1.856, \(p =\)
.064), and the no adjective utterance type and the within-category
context (\(\beta =\) 5.573, \(t =\) 1.793, \(p =\) .073). No other
effects were significant or marginally significant. Thus, participants
inferred that an object referred to in an intelligible utterance with no
description was more typical of its category on the target feature than
an object referred to with an alien utterance. Participants did not
substantially adjust their inferences based on the object context. The
marginal interactions between the within-category context and both the
adjective and no adjective utterance types suggest that people might
have judged the target feature as slightly more prevalent in the
within-category context when intelligible utterances (with a bare noun
or with an adjective) were used compared to the alien utterance. If
people are discounting their atypicality inferences when the adjective
is necessary for reference, we should expect them to have slightly
higher typicality judgments in the within-category context when an
adjective is used, and this marginal interaction suggests that this may
be the case. However, since typicality judgments in the no adjective
utterance type are also marginally greater in the within-category
context, and because judgments in the alien utterance conditions (the
reference category) also directionally move between the two context
conditions, it is hard to interpret whether this interaction supports
the idea that people are discounting their typicality judgments based on
context.

Given that interpretation of these results with respect to the alien
utterance condition can be difficult, we pre-registered a version of the
same full model excluding alien utterance trials with the no adjective
utterance type as the reference level. This model revealed a significant
effect of utterance type: participants' prevalence judgments were lower
when an adjective was used than when it was not (\(\beta =\) -8.117,
\(t =\) -3.463, \(p =\) .001). No other effects were significant. This
replicates the main effect of interest in Experiment 1: when an
adjective is used in referring to the object, participants infer that
the described feature is less typical of that object's category than
when the feature goes unmentioned. It also shows that the possibility
that people may discount their typicality judgments based on context
(suggested by the marginal interaction described above) is not supported
when we compare the adjective and no adjective utterance types directly.

As in Experiment 1, our test of whether participants' inferences are
modulated by context is potentially complicated by people making
distinct inferences when an adjective is necessary but \emph{not} used.
Thus, we additionally tested whether participants' inferences varied by
context among only \emph{noun} trials and only \emph{adjective noun}
trials, separately. Testing only \emph{noun} trials checks directly
whether people make higher typicality judgments when an adjective is
necessary but not used, compared to when it is not necessary and not
used. To check this, we fit a model on only \emph{noun} trials, with
effects of context and feature type and their interaction, as well as
random slopes by subject (not pre-registered). Participants' inferences
among only \emph{noun} trials did not significantly differ by context
(\(\beta_{within} =\) 0.087, \(t_{within} =\) 0.046, \(p_{within} =\)
.964). In the same way, we tested whether people's inferences varied by
context among only \emph{adjective noun} trials: this is a test of
context effects that could not have been caused (or masked) by people's
inferences about adjective omission. Participants' inferences among only
\emph{adjective noun} trials did not significantly differ by context
(\(\beta_{within} =\) 3.068, \(t_{within} =\) 1.696, \(p_{within} =\)
.091). Numerically, people's prevalence judgments were slightly higher
in the within-category context, but these effects were not significant.
Thus, participants' inferences did not significantly differ between
contexts, whether tested by the interaction between utterance type and
contexts or by the effect of context among only utterances with or
without an adjective.

\begin{figure}[!tb]

{\centering \includegraphics{figs/e3-wppl-plot-1} 

}

\caption{Participants' prevalence judgments in Experiment 2, with model predictions using the parameters estimated in Experiment 1. Points indicate empirical means; error bars indicate 95\% confidence intervals computed by non-parametric bootstrapping. Solid horizontal lines indicate model predictions.}\label{fig:e3-wppl-plot}
\end{figure}

\hypertarget{model-1}{%
\subsection{Model}\label{model-1}}

To validate the model we developed for Experiment 1, we compared its
estimates using the previously fit parameters to the new data from
Experiment 2. As shown in Figure \ref{fig:e3-wppl-plot}, the model
predictions were well aligned with people's prevalence judgments. In
addition, in Experiment 1, we fixed the model's prior beliefs about the
prevalence of the target object's color or size to be centered at 50\%
because the model had seen one pseudo-exemplar of the target color/size,
and one psuedo-exemplar of the non-target color/size. In Experiment 2,
we aimed to estimate this prior empirically in the alien utterance
condition, reasoning that people could only use their prior to make a
prevalence judgment (as we asked the model to do). In both the color and
size conditions, people's judgments indeed varied around 50\%, although
in the color condition they were directionally lower. This small effect
may arise from the fact that size varies on a scale with fewer nameable
points (e.g., objects can be big, medium-sized or small) whereas color
has many nameable alternatives (e.g., red, blue, green, etc.). Thus, the
results of Experiment 2 confirm the modeling assumptions we made in
estimating people's prior beliefs, and further validate the model we
developed as a good candidate model for how people simultaneously draw
inferences about speakers' intended referents and the typicality of
these referents. That is, when people think about why a speaker chose
their referring expression, they consider the context of not only
present objects, but also the broader category to which the referent
belongs.

\hypertarget{discussion-2}{%
\subsection{Discussion}\label{discussion-2}}

In Experiment 2, we replicated the main finding of interest in
Experiment 1: when a novel object's feature is described, people infer
that the feature is rarer of its category than when it goes unmentioned.
Again, this effect was consistent across both size and color adjectives,
and people did not substantially adjust this inference based on how
necessary the description was to distinguish among potential referents.
We also added an alien language condition, in which the entire referring
expression was unintelligible to participants, to probe people's priors
on feature typicality. We found that in the alien language condition,
people judged features to be roughly between the adjective utterance and
no adjective utterance conditions, and significantly different from the
no adjective utterance condition. In the alien language condition,
people's prevalence judgments were roughly around our model's prevalence
judgments (50\%) after observing the objects on each trial and before
any inferences about the utterance.

The similarity of people's prevalence judgments in the alien language
condition and the adjective condition raises the question: is this
effect driven by an atypicality inference in the adjective conditions,
or a \emph{typicality} inference when the feature is unmentioned? Our
results suggest that it is a bit of both. When someone mentions an
object without extra description, the listener can infer that its
features are likely more typical than their prior; when they use
description, they can infer that its features are likely less typical.
Because using an extra word---an adjective---is generally not thought of
as the default way to refer to something, this effect is still best
described as a contrastive inference of \emph{atypicality} when people
use description. However, the fact that people infer high typicality
when an object is referred to without description suggests that, in some
sense, there is no neutral way to refer: people will make broader
inferences about a category from even simple mentions of an object.

\hypertarget{general-discussion-1}{%
\section{General Discussion}\label{general-discussion-1}}

When we think about what someone is trying to communicate to us, we go
far beyond the literal meanings of the words they say: we make pragmatic
inferences about why they chose those particular words rather than other
words they could have used instead. In most work on pragmatic reasoning,
speakers and listeners share the same knowledge of language, and the
question of interest is whether listeners can use their knowledge of
language to learn something about the unknown state of the world. Here
we focus on an even more challenging problem: Can pragmatic inference be
used to learn about language and the world simultaneously?

In two experiments, we showed that people infer that a noted feature is
atypical of the object being referred to. Critically, people infer that
the described feature is atypical even when the descriptor is helpful
for referential disambiguation. Why do people think that the mentioned
feature is atypical even when its mention is helpful for referential
disambiguation? If people use language for multiple goals---for example,
both for reference and for description---then listeners should reason
jointly about all of the possible reasons why speakers could have used a
word. To determine what rational listeners would do in this
circumstance, we developed an extension of the Rational Speech Act
Framework that reasons both about reference and about the typical
features of categories to which objects belong. The behavior of this
model was closely aligned to the behavior we observed in people. Because
rational inference is probabilistic rather than deterministic, the
trade-off in the model is slight: descriptors still lead to atypicality
inferences even when they are helpful for referential disambiguation.
This work thus adds to the growing body of work extending the Rational
Speech Act framework from reasoning about just reference to reasoning
about other goals as well, such as inferring that speech is hyperbolic,
inferring when speakers are being polite rather than truthful, and
learning new words in ambiguous contexts (Goodman and Frank 2016; Yoon
et al. 2020; Kao et al. 2014; Frank and Goodman 2014; Bohn et al. 2021,
2022).

In considering how people may integrate inferences about typicality and
about reference, we raised two broad possibilities: (1) a
\emph{reference-first view}, whereby if an adjective was necessary for
reference it would block an inference of atypicality completely, and (2)
a \emph{probabilistic weighing view}, whereby the goals of being
informative with respect to reference and with respect to the category
would trade off in a graded way. That is, we aimed to test whether there
was a strong trade-off or a weak trade-off. People's behavior in our
tasks is inconsistent with the reference-first view: inferences of
atypicality were not blocked when an adjective was necessary for
reference. On the other hand, our model implements the latter view and
fits the data well, but we do not find significant evidence of a
trade-off in our statistical tests of people's responses: the data are
also compatible with there being no trade-off whatsoever.

Our experiments use a particular kind of task context: alien fruits,
spoken about by alien interlocutors. Would these effects generalize
beyond these particular items, and this particular task? It is possible
that people hold expectations about how the features of fruit are
distributed---for instance, that they have stereotypical colors. These
overhypotheses about how basic-level categories' features are
distributed within a superordinate category (Kemp, Perfors, and
Tenenbaum 2007) may make people's inferences about fruit different from
their inferences about other superordinate categories. More broadly,
people may make different kinds of inferences in more naturalistic
communicative settings. In our task, people were asked to make several
typicality judgments, which may have encouraged them to focus on how the
aliens' utterances could help them learn about the world rather than
focusing on other communicative goals such as reference. It is possible
that people's inferences would reflect a clear tradeoff between
reference and communicating atypicality if reference was a more salient
communicative goal in the task. Further, it may be easier to attribute
nuanced communicative goals to \emph{people} talking about plausibly
real things, rather than to alien characters. So, though we find people
do use pragmatic inferences to learn about new categories in these
artificial tasks, these inferences may play out differently in more
naturalistic contexts with more communicative goals plausibly in play.

In Chapter 1, we established that people tend to mention atypical rather
than typical features. In this chapter, we showed that adults make
appropriate pragmatic inferences given how speakers describe: they infer
that a mentioned feature is likely to be less typical of the mentioned
category. However, the ability to learn about new categories using
contrastive inference most obviously serves budding language
learners---children. To fully appreciate the potential of these
inferences to allow people to learn about the world, we must study their
development, which we will turn to in Chapter 3.

\hypertarget{how-children-use-contrastive-inference-to-learn-about-new-categories}{%
\chapter{How children use contrastive inference to learn about new
categories}\label{how-children-use-contrastive-inference-to-learn-about-new-categories}}

The speech children hear mentions more atypical than typical features.
Depending on children's pragmatic abilities, this input could provide
helpful information or pose a misleading challenge as children learn
about the world. If children are able to make the contrastive inference
that description tends to pick out atypical features, they could use
description to go beyond learning about what they directly experience.
If, on the other hand, they merely associate the mentioned feature with
the mentioned category, they may mistakenly learn that atypical features
are more common than they actually are.

In general, children's pragmatic abilities are thought to undergo
prolonged development, not reaching adult-like performance until well
into schooling age. The most thoroughly studied pragmatic inference in
children, scalar implicature, tells a bleak story about children's
ability to make pragmatic inferences at a young age. Scalar implicature
is the phenomenon in which use of a weak scalar term (`some,' `might')
implies that a stronger scalar term (`all,' `must') is not true---for
example, ``I ate some of the cookies'' implies I did not eat all of
them. This inference can be derived by reasoning that had the speaker
meant the stronger meaning, they would have used the stronger term.
Adults consistently interpret the word `some' to mean `some but not
all,' rating the use of `some' as unnatural when `all' is applicable and
taking longer to respond to such instances (Bott and Noveck 2004; Degen
and Tanenhaus 2015). Until at least the age of 5 and in some tasks up to
10 years old, children fail to limit the use of `some' in this way,
accepting `some' as a descriptor when `all' is true (Noveck 2001;
Papafragou and Musolino 2003). This deficit is found in a range of
measures, from acceptability judgments to eye-tracking (Huang and
Snedeker 2009). Later work has found that children likely lack this
ability because they fail to activate alternative descriptions, so
cannot reason that the speaker should have said `all' and not `some' if
all is true (Barner, Brooks, and Bale 2011), and because they lack a
meta-understanding of these tasks (Papafragou and Musolino 2003). When
given supportive context, like named alternatives or training on the
task, 4- and 5-year-olds improve at these implicatures (Barner, Brooks,
and Bale 2011; Papafragou and Musolino 2003; Foppolo, Guasti, and
Chierchia 2012). However, across experiments, performance on scalar
implicature remains fragile well into school age.

Contrastive inference from description, however, may be a more
accessible form of pragmatic inference because the relevant alternatives
are more easily accessible. In the case of using contrastive inference
to resolve reference (e.g., ``the tall\ldots{}'' prompts looking to a
tall object with a shorter counterpart), the relevant alternatives are
available in the environment. By the age of 5, children can use
contrastive inferences to direct their attention among familiar present
objects (Huang and Snedeker 2008), and when given extra time to orient
to the referent, show budding abilities by the age of 3 (Davies et al.
2021). Description paired with other contrastive cues can allow children
to restrict reference among novel objects or objects with novel
properties, though imperfectly (Gelman and Markman 1985; Diesendruck,
Hall, and Graham 2006).

What about when the contrasting set is not available in the environment,
but is the referent's category? Preliminary evidence also suggests that
contrastive inference about typicality may be possible for young
children. When paired with other contrastive cues, 4-year-olds can make
inferences about novel object typicality, reasoning that ``the TALL
zib'' suggests other zibs are generally shorter (Horowitz and Frank
2016). This work provided a useful demonstration that adjective use can
contribute to inferences about feature typicality, though it did not
isolate the effect of adjectives specifically. Their experiments used
several contrastive cues, such as prosody (contrastive stress on the
adjective: ``TALL zib''), demonstrative phrases that may have marked the
object as unique (``this one'') and expressions of surprise at the
object (``wow''), and participants may have inferred the object was
atypical primarily from these cues and not from the adjective. Further,
these experiments used a forced-choice measure that does not allow a
precise estimate of how much children's typicality judgments shift from
adjective use. Thus, in this experiment, we set out to develop a task
that would isolate the effect of adjective use and measure children's
typicality judgments in a more graded way.

In this chapter, we report an exploratory study of children's abilities
to make contrastive inferences about typicality. To do this, we used a
task similar to those done by adults in Chapter 2, having children
observe novel categories and make inferences about the typicality of
their features. We study 5- to 6-year-old children, an age at which key
pragmatic abilities are developing and when children can use contrastive
inferences to direct their attention among familiar referents. Because
children at this age struggle to explicitly reason about and report
proportions (see Boyer, Levine, and Huttenlocher 2008 for a review), we
will have children report their typicality judgments with the help of
visual depictions of \emph{few}, \emph{some}, \emph{most}, and
\emph{almost all} objects having a feature. The purpose of this
exploratory study is both to see whether children can make sensible
responses on this measure and to gather preliminary evidence about
children's contrastive inferences.

\hypertarget{method-3}{%
\section{Method}\label{method-3}}

\hypertarget{participants.-2}{%
\subsection{Participants.}\label{participants.-2}}

We recruited 30 5--6-year-old children raised with 90\% or greater
English language exposure to participate in this task. Children were
recruited from a database with mostly families living in the Chicago
area, and some families living elsewhere in the United States, and the
study was conducted remotely on Zoom. Data from one participant was
excluded due to connection difficulties in the call. In the final
sample, 15 5-year-olds and 14 6-year-olds participated.

\begin{figure}
\centering
\includegraphics{figs/kid-trial-1.pdf}
\caption{An example of the novel objects shown on a trial. In each
trial, two objects of the same shape and differing on the critical
feature were shown sequentially. In adjective noun trials, the critical
feature was mentioned for the object that had it (e.g., the wide toma
was called a ``wide toma'') and in noun trials, no features were
mentioned (e.g., both tomas were just called a ``toma''.}
\end{figure}

\begin{figure}
\centering
\includegraphics{figs/kid-dv-1.pdf}
\caption{An example of the prevalence judgment children were asked to
make. Children chose between clouds of novel objects representing few,
some, most, and almost all of the novel category having the feature. The
experimenter asked, e.g., ``Let's think about all of the blickets on
this planet. How many blickets do you think are spotted? Few of the
blickets, some of the blickets, most of the blickets, or almost all of
the blickets?''}
\end{figure}

\hypertarget{design-and-procedure.}{%
\subsection{Design and Procedure.}\label{design-and-procedure.}}

Children participated in a novel object learning task in which they
observed novel objects and made inferences about them. They were
introduced to an alien named Blip, who would show things from her
planet. Blip's utterances were presented both as recorded audio and
displayed in a text bubble on the screen. In each trial, Blip first said
``Let's see what I have\ldots{}'' and then sequentially showed two
objects with the same name and shape. The two objects differed on the
critical feature. In \emph{adjective noun} trials, the critical feature
was mentioned (e.g., one object was labeled ``It's a blicket'' and the
other was labeled ``It's a striped blicket''); in \emph{noun} trials,
the critical feature was not mentioned (e.g., one object was labeled
``It's a blicket'' and the other was also labeled ``It's a blicket'').

After each trial, children were asked to make a judgment about the
prevalence of the critical feature in the novel category. For instance,
they were asked, ``Let's think about all of the blickets on this planet.
How many blickets do you think are spotted?'' There were four options on
the screen, each a cloud of six of the same shape of novel object, with
differing proportions having the critical feature and in color (and the
remaining objects without the feature and in grey). The options were
\emph{Few} (1/6 with feature), \emph{Some} (2/6 with feature),
\emph{Most} (4/6 with feature), and \emph{Almost All} (5/6 with
feature). After asking the question, the experimenter said the options:
``Few of the blickets, some of the blickets, most of the blickets, or
almost all of the blickets?'' Children responded verbally. If they
paused or seemed uncertain, the experimenter repeated the options. If
the child preferred to point to the option on the screen (as happened
with one participant), the experimenter asked the child's parent to
report the option they pointed to.

There were six trials in total. Half of trials were \emph{adjective
noun} trials and half were \emph{noun} trials, and this factor was
crossed with the feature type: size (wide or tall), color (blue or red),
and pattern (spotted or striped). At each trial, the novel object shape
and novel object name were randomly assigned out of a set of six names
(modi, blicket, wug, toma, gade, or sprock) and shapes. The ordering of
two objects in each trial (one with the critical feature and one
without) was random.

Before the main task, children did two practice trials with familiar
objects to establish that they understood the response measure. The two
practice questions were: ``Let's think about all of the cookies in the
world. How many cookies do you think are square?'' and ``Let's think
about all of the bananas in the world. How many bananas do you think are
yellow?'' They responded on the same scale used in the main task trials.

\hypertarget{performance-on-practice-trials}{%
\section{Performance on practice
trials}\label{performance-on-practice-trials}}

Children's performance on the two practice trials with familiar objects
can help give us a sense of whether they understand the typicality
measure in this task. If the children understand this measure, we expect
them to report that bananas are more commonly yellow than cookies are
square. Out of 29 participants, 15 (0.517\%) rated bananas to be more
commonly yellow than cookies are square (0.4\% of 5-year-olds and
0.643\% of 6-year-olds). That is, many children, especially the
5-year-olds, either did not understand this measure well or did not
believe that cookies are not typically square and bananas are typically
yellow. Below, we will report the results of the main task both for all
children and, separately, for just the children who performed correctly
on the familiar practice trials to see whether there is evidence for
contrastive inference among children who understood the measure.

\begin{figure}[!tb]

{\centering \includegraphics{figs/kid-results-1} 

}

\caption{Children's prevalence judgments across utterance conditions and feature types.}\label{fig:kid-results}
\end{figure}

\hypertarget{results-4}{%
\section{Results}\label{results-4}}

Our key question is whether children make different inferences when an
object's feature is mentioned than when it is not. To test this
question, we fit a linear regression with children's prevalence choices
as the outcome (coded as \emph{few} = 1, \emph{some} = 2, \emph{most} =
3, and \emph{almost all} = 4) and utterance type (\emph{noun}
vs.~\emph{adjective noun}), feature type (color, size, or pattern), and
their interaction as predictors, as well as a random intercept by
subject. The effect of utterance type was marginally significant:
children's prevalence judgments were marginally lower when there was an
adjective in the utterance (\(\beta =\) -0.552, \(t =\) -1.951, \(p =\)
0.053). Effects of feature type were not significant
(\(\beta_{pattern} =\) -0.448, \(t =\) -1.585, \(p =\) 0.115;
\(\beta_{size} =\) -0.276, \(t =\) -0.975, \(p =\) 0.331), nor were
interactions between utterance type and feature type
(\(\beta_{adjective-noun*pattern} =\) 0.655, \(t =\) 1.638, \(p =\)
0.104; \(\beta_{adjective-noun*size} =\) 0.483, \(t =\) 1.207, \(p =\)
0.23). Though effects of feature type and the interaction between
utterance type and feature type are not significant, visually examining
the plotted data, the overall marginal effect of utterance type seems to
be driven by the color condition. Overall, we find weak evidence that
children infer that mentioned features are less typical. Children's
prevalence judgments are shown in Figure \ref{fig:kid-results}.

\begin{figure}[!tb]

{\centering \includegraphics{figs/kid-results-measure-1} 

}

\caption{Prevalence judgments among only children who answered the practice trials correctly. These children rate features to be less prevalent when they are mentioned with an adjective.}\label{fig:kid-results-measure}
\end{figure}

Based on their performance in the practice trials, it seems that many
children did not understand the prevalence measure well. We can
separately test the performance of children who correctly answered the
practice trials to see whether children who understand the measure
demonstrate contrastive inference. We fit the same model specification
to only children who rated bananas to be yellow more typically than
cookies are square. Among these children, there is a significant effect
of utterance type, such that they infer that mentioned features are less
typical (\(\beta =\) -1.133, \(t =\) -3.069, \(p =\) .003). Effects of
feature type were not significant (\(\beta_{pattern} =\) -0.467, \(t =\)
-1.264, \(p =\) 0.21; \(\beta_{size} =\) -0.2, \(t =\) -0.542, \(p =\)
0.59), nor were interactions between utterance type and feature type
(\(\beta_{adjective-noun*pattern} =\) 0.867, \(t =\) 1.66, \(p =\)
0.101; \(\beta_{adjective-noun*size} =\) 0.933, \(t =\) 1.787, \(p =\)
0.078). The utterance effect is also directionally present across all
three feature conditions. The small sample of children who performed
correctly on familiar trials judged mentioned features to be less
typical (Figure \ref{fig:kid-results-measure}).

\hypertarget{discussion-3}{%
\section{Discussion}\label{discussion-3}}

In this chapter, we asked how children develop the inference that when a
feature of a novel category is mentioned, that feature is likely to be
atypical of the category. One possibility is that children simply
associate the words, features and categories that are salient in an
instance of reference. This would lead children to think a mentioned
feature is representative or typical of the mentioned category. Another
possibility is that children make the kind of contrastive inference
adults make, inferring that the mentioned feature is atypical. In an
exploratory study, we found suggestive evidence that 5--6-year-old
children are not making an associative inference, and are directionally
making a contrastive inference about mentioned features. Further, the
small sample of children who performed correctly on practice trials with
familiar objects made significantly lower typicality judgments about
mentioned features. However, judging typicality is difficult for young
children, and participants struggled with our measure overall. Evidence
from this task is only preliminary, and calls for confirmatory tests
with larger sample sizes and for the development of measures that are
more sensible for young children.

\hypertarget{conclusion}{%
\chapter*{Conclusion}\label{conclusion}}
\addcontentsline{toc}{chapter}{Conclusion}

This dissertation examines how speakers selectively describe remarkable
features and how listeners use this selective description to learn more
about the world. In doing so, it inverts the framework that has
positioned pragmatic inference as augmenting literal meaning that is
already known, instead considering how people can use pragmatics to
learn more about the semantics of unfamiliar things.

To understand how people use description to learn about the world, we
first must know how description is used. Chapter 1 illustrates how
caregivers use description in speaking to children, as well as
establishing how adults use description when speaking to other adults
and how children themselves use description. We find that parents
predominantly mention atypical rather than typical features when
speaking to children, as do adults when speaking to other adults.

We also examined how children themselves use description, and found that
they mostly talk about the atypical features of things. There are
several language-generating processes that may explain children's use of
description. One possibility is that children understand that
description is used to draw a distinction between the described thing
and some relevant alternatives---that they are using description
informatively to highlight atypicality. Another possibility is that
children are broadly reflecting the distribution of adjective-noun usage
in their parents' speech, simply by producing the kinds of
adjective-noun pairs they have heard before. A third is that their
pattern of description is largely explained by local mimicry---that
children are directly repeating back adjectives and nouns their parents
used recently in conversation. More focused corpus analyses, as well as
experiments eliciting children's adjective production, are necessary to
distinguish between these possibilities.

The pattern of description we find in parents' speech to children is
consistent with the idea that people use language informatively with
relation to background knowledge of the world, rather than giving
veridical running commentary on the world's features. This finding
raises questions about how children use description to learn, given that
so many accounts of language learning rest on children forming
associations among co-occurring words, features, and concepts. To test
what kind of typicality information is derivable from language alone, we
investigated whether language models that use associative learning among
words can extract typical feature information from language. We find
that simpler distributional semantics models do poorly in distinguishing
between the typical and atypical features of nouns, with implications
both for associative accounts of children's language learning and for
language modeling. However, a large language model with a more complex
architecture and access to more and different language input than
children receive---GPT-3---was able to capture adjective-noun typicality
fairly well. Overall, our findings highlight the complexity of learning
about the world from language that describes it selectively.

However, perhaps people---unlike simpler associative language
models---know that language is used to selectively remark on the world,
and can use this fact to learn about the unfamiliar. In Chapter 2, we
investigated how adults make inferences about novel object categories,
and found that they can use description to infer that a described
feature is atypical. Further, even when description may have been used
for another purpose---to establish reference---people make inferences
about typicality. We find that a model that considers the utility of
utterances with respect to reference and typicality captures people's
inferences. Much prior work has only considered the use of description
in distinguishing between present referents (Pechmann 1989; Engelhardt,
Barış Demiral, and Ferreira 2011; Mangold and Pobel 1988), and even work
that has incorporated typicality has focused on reference as the primary
goal of description (Sedivy 2003; Mitchell, Reiter, and Deemter 2013;
Westerbeek, Koolen, and Maes 2015; Rubio-Fernández 2016). Our findings
emphasize that conveying typicality is likely a central factor in
referring, and inferences about typicality are not secondary to or
blocked by the purpose of establishing reference. Further, though
pragmatics is generally conceived of as a layer of meaning that only
emerges on top of a more stable semantics, our findings demonstrate the
reverse: when semantic meaning is uncertain, people can use pragmatics
to resolve it.

The ability to exploit description to learn more about the world than
one has observed directly is most useful to people who are still rapidly
learning---children. In Chapter 3, we investigated how 5- to 6-year-old
children make contrastive inferences about typicality. The results of
our preliminary experiment show that it is difficult to elicit graded
typicality judgments from children. However, children who understand our
typicality measure do not make associative inferences in this task;
rather, we find preliminary evidence that these children directionally
make contrastive inferences. Taken together with evidence from our
corpus analysis, this preliminary study suggests that by the age of 5 or
6 children are \emph{not} making associative inferences about the
atypical adjective-noun pairs they hear, and may be making contrastive
inferences instead. However, further work with better measures is
necessary to confirm this finding and examine how younger children
interpret the description they hear.

The core computation in pragmatic inference is reasoning about
alternatives---things the speaker could have said and did not. Given
that others are reasoning about these alternatives, no choice is
neutral. In the studies in Chapter 2, for instance, using an adjective
in referring to an object led people to infer that the feature described
by that adjective was less typical than if it had not been mentioned.
But, conversely, \emph{not} using an adjective led them to think that
the feature was more typical than if they could not understand the
meaning of the utterance at all---all communicative choices leak one's
beliefs about the world. This has implications not only for learning
about novel concrete objects, as people did here, but for learning about
less directly accessible entities such as abstract concepts and social
groups. These inferences can be framed positively, as ways for learners
to extract additional knowledge that was not directly conveyed, but can
also spread beliefs that the speaker does not intend. The principle that
people speak informatively is simple, but it holds unintuitive
consequences---among speakers and listeners, humans and machines, adults
and children---for describing and learning about the world.

\newpage

\hypertarget{references}{%
\chapter*{References}\label{references}}
\addcontentsline{toc}{chapter}{References}

\hypertarget{refs}{}
\begin{cslreferences}
\leavevmode\hypertarget{ref-akhtar_role_1996}{}%
Akhtar, Nameera, Malinda Carpenter, and Michael Tomasello. 1996. ``The
Role of Discourse Novelty in Early Word Learning.'' \emph{Child
Development} 67 (2): 635--45.
\url{https://doi.org/10.1111/j.1467-8624.1996.tb01756.x}.

\leavevmode\hypertarget{ref-albert_cabnc_2015}{}%
Albert, Saul, Laura E. de Ruiter, and J. P. de Ruiter. 2015. ``CABNC:
The Jeffersonian Transcription of the Spoken British National Corpus.''

\leavevmode\hypertarget{ref-aparicio2016processing}{}%
Aparicio, Helena, Ming Xiang, and Christopher Kennedy. 2016.
``Processing Gradable Adjectives in Context: A Visual World Study.'' In
\emph{Semantics and Linguistic Theory}, 25:413--32.

\leavevmode\hypertarget{ref-arts_overspecification_2011}{}%
Arts, Anja, Alfons Maes, Leo G. M. Noordman, and Carel Jansen. 2011.
``Overspecification in Written Instruction.'' \emph{Linguistics} 49 (3):
555--74.

\leavevmode\hypertarget{ref-baillargeon1994}{}%
Baillargeon, Renee. 1994. ``How Do Infants Learn About the Physical
World?'' \emph{Current Directions in Psychological Science} 3 (5):
133--40.

\leavevmode\hypertarget{ref-baker1988}{}%
Baker, Nancy D., and Patricia M. Greenfield. 1988. ``The Development of
New and Old Information in Young Children's Early Language.''
\emph{Language Sciences} 10 (1): 3--34.

\leavevmode\hypertarget{ref-barner_accessing_2011}{}%
Barner, David, Neon Brooks, and Alan Bale. 2011. ``Accessing the Unsaid:
The Role of Scalar Alternatives in Children's Pragmatic Inference.''
\emph{Cognition} 118 (1): 84--93.

\leavevmode\hypertarget{ref-bedny2019}{}%
Bedny, Marina, Jorie Koster-Hale, Giulia Elli, Lindsay Yazzolino, and
Rebecca Saxe. 2019. ``There's More to `Sparkle' Than Meets the Eye:
Knowledge of Vision and Light Verbs Among Congenitally Blind and Sighted
Individuals.'' \emph{Cognition} 189: 105--15.

\leavevmode\hypertarget{ref-bergey_morris_2020}{}%
Bergey, Claire, Benjamin Morris, and Daniel Yurovsky. 2020. ``Children
Hear More About What Is Atypical Than What Is Typical.'' PsyArXiv.
\url{https://doi.org/10.31234/osf.io/5wvu8}.

\leavevmode\hypertarget{ref-bohn_predicting_2022}{}%
Bohn, Manuel, Michael Henry Tessler, Megan Merrick, and Michael C.
Frank. 2022. ``Predicting Pragmatic Cue Integration in Adults' and
Children's Inferences About Novel Word Meanings.'' \emph{Journal of
Experimental Psychology: General} 151: 2927--42.
\url{https://doi.org/10.1037/xge0001216}.

\leavevmode\hypertarget{ref-bohn_how_2021}{}%
---------. 2021. ``How Young Children Integrate Information Sources to
Infer the Meaning of Words.'' \emph{Nature Human Behaviour} 5 (8):
1046--54. \url{https://doi.org/10.1038/s41562-021-01145-1}.

\leavevmode\hypertarget{ref-bott_utterances_2004}{}%
Bott, Lewis, and Ira A. Noveck. 2004. ``Some Utterances Are
Underinformative: The Onset and Time Course of Scalar Inferences.''
\emph{Journal of Memory and Language} 51 (3): 437--57.

\leavevmode\hypertarget{ref-boyer_development_2008}{}%
Boyer, Ty W., Susan C. Levine, and Janellen Huttenlocher. 2008.
``Development of Proportional Reasoning: Where Young Children Go
Wrong.'' \emph{Developmental Psychology} 44: 1478--90.
\url{https://doi.org/10.1037/a0013110}.

\leavevmode\hypertarget{ref-brown_language_2020}{}%
Brown, Tom B., Benjamin Mann, Nick Ryder, Melanie Subbiah, Jared Kaplan,
Prafulla Dhariwal, Arvind Neelakantan, et al. 2020. ``Language Models
Are Few-Shot Learners.'' arXiv.
\url{https://doi.org/10.48550/arXiv.2005.14165}.

\leavevmode\hypertarget{ref-brysbaert2014}{}%
Brysbaert, Marc, Amy Beth Warriner, and Victor Kuperman. 2014.
``Concreteness Ratings for 40 Thousand Generally Known English Word
Lemmas.'' \emph{Behavior Research Methods} 46 (3): 904--11.

\leavevmode\hypertarget{ref-clark_pragmatics_1990}{}%
Clark, Eve V. 1990. ``On the Pragmatics of Contrast.'' \emph{Journal of
Child Language} 17 (2): 417--31.
\url{https://doi.org/10.1017/S0305000900013842}.

\leavevmode\hypertarget{ref-coleman_audio_2012}{}%
Coleman, John, Ladan Baghai-Ravary, John Pybus, and Sergio Grau. 2012.
``Audio BNC: The Audio Edition of the Spoken British National Corpus.''

\leavevmode\hypertarget{ref-dale_graded_2007}{}%
Dale, Rick, Caitlin Kehoe, and Michael J Spivey. 2007. ``Graded Motor
Responses in the Time Course of Categorizing Atypical Exemplars.''
\emph{Memory \& Cognition} 35 (1): 15--28.

\leavevmode\hypertarget{ref-davies_three-year-olds_2021}{}%
Davies, Catherine, Jamie Lingwood, Bissera Ivanova, and Sudha
Arunachalam. 2021. ``Three-Year-Olds' Comprehension of Contrastive and
Descriptive Adjectives: Evidence for Contrastive Inference.''
\emph{Cognition} 212 (July): 104707.
\url{https://doi.org/10.1016/j.cognition.2021.104707}.

\leavevmode\hypertarget{ref-degen_when_2020}{}%
Degen, Judith, Robert D Hawkins, Caroline Graf, Elisa Kreiss, and Noah D
Goodman. 2020. ``When Redundancy Is Useful: A Bayesian Approach to
`Overinformative' Referring Expressions.'' \emph{Psychological Review}
127: 591--621.

\leavevmode\hypertarget{ref-degen_processing_2015}{}%
Degen, Judith, and Michael K. Tanenhaus. 2015. ``Processing Scalar
Implicature: A Constraint-Based Approach.'' \emph{Cognitive Science} 39
(4): 667--710.

\leavevmode\hypertarget{ref-devlin2018}{}%
Devlin, Jacob, Ming-Wei Chang, Kenton Lee, and Kristina Toutanova. 2018.
``Bert: Pre-Training of Deep Bidirectional Transformers for Language
Understanding.'' \emph{arXiv Preprint arXiv:1810.04805}.

\leavevmode\hypertarget{ref-diesendruck_childrens_2006}{}%
Diesendruck, Gil, D. Geoffrey Hall, and Susan A. Graham. 2006.
``Children's Use of Syntactic and Pragmatic Knowledge in the
Interpretation of Novel Adjectives.'' \emph{Child Development} 77 (1):
16--30.

\leavevmode\hypertarget{ref-engelhardt_over-specified_2011}{}%
Engelhardt, Paul E., Ş. Barış Demiral, and Fernanda Ferreira. 2011.
``Over-Specified Referring Expressions Impair Comprehension: An ERP
Study.'' \emph{Brain and Cognition}, Special Section: Aggression and
peer victimization: Genetic, neurophysiological, and neuroendocrine
considerations, 77 (2): 304--14.
\url{https://doi.org/10.1016/j.bandc.2011.07.004}.

\leavevmode\hypertarget{ref-firth1957}{}%
Firth, John R. 1957. ``A Synopsis of Linguistic Theory, 1930-1955.''
\emph{Studies in Linguistic Analysis}.

\leavevmode\hypertarget{ref-foppolo_scalar_2012}{}%
Foppolo, Francesca, Maria Teresa Guasti, and Gennaro Chierchia. 2012.
``Scalar Implicatures in Child Language: Give Children a Chance.''
\emph{Language Learning and Development} 8 (4): 365--94.

\leavevmode\hypertarget{ref-frank2012}{}%
Frank, Michael C, and Noah D Goodman. 2012. ``Predicting Pragmatic
Reasoning in Language Games.'' \emph{Science} 336 (6084): 998--98.

\leavevmode\hypertarget{ref-frank2014}{}%
---------. 2014. ``Inferring Word Meanings by Assuming That Speakers Are
Informative.'' \emph{Cognitive Psychology} 75: 80--96.

\leavevmode\hypertarget{ref-frank2009}{}%
Frank, Michael C, Noah D Goodman, and Joshua B Tenenbaum. 2009. ``Using
Speakers' Referential Intentions to Model Early Cross-Situational Word
Learning.'' \emph{Psychological Science} 20 (5): 578--85.

\leavevmode\hypertarget{ref-gelman_implicit_1985}{}%
Gelman, Susan A., and Ellen M. Markman. 1985. ``Implicit Contrast in
Adjectives Vs. Nouns: Implications for Word-Learning in Preschoolers*.''
\emph{Journal of Child Language} 12 (1): 125--43.

\leavevmode\hypertarget{ref-goldin-meadow2014}{}%
Goldin-Meadow, Susan, Susan C Levine, Larry V Hedges, Janellen
Huttenlocher, Stephen W Raudenbush, and Steven L Small. 2014. ``New
Evidence About Language and Cognitive Development Based on a
Longitudinal Study: Hypotheses for Intervention.'' \emph{American
Psychologist} 69 (6): 588.

\leavevmode\hypertarget{ref-goodman2016}{}%
Goodman, Noah D, and Michael C Frank. 2016. ``Pragmatic Language
Interpretation as Probabilistic Inference.'' \emph{Trends in Cognitive
Sciences} 20 (11): 818--29.

\leavevmode\hypertarget{ref-grice1975logic}{}%
Grice, H Paul. 1975. ``Logic and Conversation.'' \emph{1975}, 41--58.

\leavevmode\hypertarget{ref-harris2006}{}%
Harris, Paul L, and Melissa A Koenig. 2006. ``Trust in Testimony: How
Children Learn About Science and Religion.'' \emph{Child Development} 77
(3): 505--24.

\leavevmode\hypertarget{ref-horowitz_childrens_2016}{}%
Horowitz, Alexandra C., and Michael C. Frank. 2016. ``Children's
Pragmatic Inferences as a Route for Learning About the World.''
\emph{Child Development} 87 (3): 807--19.

\leavevmode\hypertarget{ref-huang_semantic_2009}{}%
Huang, Yi Ting, and Jesse Snedeker. 2009. ``Semantic Meaning and
Pragmatic Interpretation in 5-Year-Olds: Evidence from Real-Time Spoken
Language Comprehension.'' \emph{Developmental Psychology} 45 (6):
1723--39.

\leavevmode\hypertarget{ref-huangsnedeker2008}{}%
---------. 2008. ``Use of Referential Context in Children's Language
Processing.'' \emph{Proceedings of the 30th Annual Meeting of the
Cognitive Science Society}, January.

\leavevmode\hypertarget{ref-johns2012}{}%
Johns, Brendan T, and Michael N Jones. 2012. ``Perceptual Inference
Through Global Lexical Similarity.'' \emph{Topics in Cognitive Science}
4 (1): 103--20.

\leavevmode\hypertarget{ref-kanwisher}{}%
Kanwisher, Nancy, Roger P. Woods, Marco Iacoboni, and John C. Mazziotta.
1997. ``A Locus in Human Extrastriate Cortex for Visual Shape
Analysis.'' \emph{Journal of Cognitive Neuroscience} 9 (1): 133--42.

\leavevmode\hypertarget{ref-kao2014}{}%
Kao, Justine T, Jean Y Wu, Leon Bergen, and Noah D Goodman. 2014.
``Nonliteral Understanding of Number Words.'' \emph{Proceedings of the
National Academy of Sciences} 111 (33): 12002--7.

\leavevmode\hypertarget{ref-kemp_learning_2007}{}%
Kemp, Charles, Andrew Perfors, and Joshua B. Tenenbaum. 2007. ``Learning
Overhypotheses with Hierarchical Bayesian Models.'' \emph{Developmental
Science} 10 (3): 307--21.
\url{https://doi.org/10.1111/j.1467-7687.2007.00585.x}.

\leavevmode\hypertarget{ref-kennedy_vagueness_2007}{}%
Kennedy, Christopher. 2007. ``Vagueness and Grammar: The Semantics of
Relative and Absolute Gradable Adjectives.'' \emph{Linguistics and
Philosophy} 30 (1): 1--45.
\url{https://doi.org/10.1007/s10988-006-9008-0}.

\leavevmode\hypertarget{ref-kim_knowledge_2019}{}%
Kim, Judy S., Giulia V. Elli, and Marina Bedny. 2019. ``Knowledge of
Animal Appearance Among Sighted and Blind Adults.'' \emph{Proceedings of
the National Academy of Sciences} 116 (23): 11213--22.
\url{https://doi.org/10.1073/pnas.1900952116}.

\leavevmode\hypertarget{ref-landau2009}{}%
Landau, Barbara, Lila R Gleitman, and Barbara Landau. 2009.
\emph{Language and Experience: Evidence from the Blind Child}. Vol. 8.
Harvard University Press.

\leavevmode\hypertarget{ref-landauer1997}{}%
Landauer, Thomas K, and Susan T Dumais. 1997. ``A Solution to Plato's
Problem: The Latent Semantic Analysis Theory of Acquisition, Induction,
and Representation of Knowledge.'' \emph{Psychological Review} 104 (2):
211.

\leavevmode\hypertarget{ref-legare2016}{}%
Legare, Cristine H, and Paul L Harris. 2016. ``The Ontogeny of Cultural
Learning.'' \emph{Child Development} 87 (3): 633--42.

\leavevmode\hypertarget{ref-lewis2019}{}%
Lewis, Molly, Martin Zettersten, and Gary Lupyan. 2019. ``Distributional
Semantics as a Source of Visual Knowledge.'' \emph{Proceedings of the
National Academy of Sciences} 116 (39): 19237--8.

\leavevmode\hypertarget{ref-mangold_informativeness_1988}{}%
Mangold, Roland, and Rupert Pobel. 1988. ``Informativeness and
Instrumentality in Referential Communication.'' \emph{Journal of
Language and Social Psychology} 7 (3-4): 181--91.

\leavevmode\hypertarget{ref-mikolov2018}{}%
Mikolov, Tomas, Edouard Grave, Piotr Bojanowski, Christian Puhrsch, and
Armand Joulin. 2018. ``Advances in Pre-Training Distributed Word
Representations.'' In \emph{Proceedings of the International Conference
on Language Resources and Evaluation (Lrec 2018)}.

\leavevmode\hypertarget{ref-mikolov2013}{}%
Mikolov, Tomas, Ilya Sutskever, Kai Chen, Greg S Corrado, and Jeff Dean.
2013. ``Distributed Representations of Words and Phrases and Their
Compositionality.'' In \emph{Advances in Neural Information Processing
Systems}, 3111--9.

\leavevmode\hypertarget{ref-mitchell_2013}{}%
Mitchell, Margaret, Ehud Reiter, and Kees van Deemter. 2013.
``Typicality and Object Reference,'' 7.

\leavevmode\hypertarget{ref-nietal}{}%
Ni, Weijia. 1996. ``Sidestepping Garden Paths: Assessing the
Contributions of Syntax, Semantics and Plausibility in Resolving
Ambiguities.'' \emph{Language and Cognitive Processes} 11 (3): 283--334.

\leavevmode\hypertarget{ref-noveck_when_2001}{}%
Noveck, Ira A. 2001. ``When Children Are More Logical Than Adults:
Experimental Investigations of Scalar Implicature.'' \emph{Cognition} 78
(2): 165--88.

\leavevmode\hypertarget{ref-papafragou_scalar_2003}{}%
Papafragou, Anna, and Julien Musolino. 2003. ``Scalar Implicatures:
Experiments at the Semantics--Pragmatics Interface.'' \emph{Cognition}
86 (3): 253--82.

\leavevmode\hypertarget{ref-pechmann_incremental_1989}{}%
Pechmann, Thomas. 1989. ``Incremental Speech Production and Referential
Overspecification.'' \emph{Linguistics} 27 (1): 89--110.

\leavevmode\hypertarget{ref-rhodes2012}{}%
Rhodes, Marjorie, Sarah-Jane Leslie, and Christina M Tworek. 2012.
``Cultural Transmission of Social Essentialism.'' \emph{Proceedings of
the National Academy of Sciences} 109 (34): 13526--31.

\leavevmode\hypertarget{ref-rogers2004}{}%
Rogers, Timothy T, and James L McClelland. 2004. \emph{Semantic
Cognition: A Parallel Distributed Processing Approach}. MIT press.

\leavevmode\hypertarget{ref-rosch_structural_1976}{}%
Rosch, Eleanor, Carol Simpson, and R Scott Miller. 1976. ``Structural
Bases of Typicality Effects.'' \emph{Journal of Experimental Psychology:
Human Perception and Performance} 2 (4): 491.

\leavevmode\hypertarget{ref-rubio-fernandez_how_2016}{}%
Rubio-Fernández, Paula. 2016. ``How Redundant Are Redundant Color
Adjectives? An Efficiency-Based Analysis of Color Overspecification.''
\emph{Frontiers in Psychology} 7.

\leavevmode\hypertarget{ref-ryskin2019information}{}%
Ryskin, Rachel, Chigusa Kurumada, and Sarah Brown-Schmidt. 2019.
``Information Integration in Modulation of Pragmatic Inferences During
Online Language Comprehension.'' \emph{Cognitive Science} 43 (8):
e12769.

\leavevmode\hypertarget{ref-rehurek2010}{}%
Řehůřek, Radim, and Petr Sojka. 2010. ``Software Framework for Topic
Modelling with Large Corpora.'' In \emph{Proceedings of the LREC 2010
Workshop on New Challenges for NLP Frameworks}, 45--50. Valletta, Malta:
ELRA.

\leavevmode\hypertarget{ref-savic_exposure_2022}{}%
Savic, Olivera, Layla Unger, and Vladimir M. Sloutsky. 2022. ``Exposure
to Co-Occurrence Regularities in Language Drives Semantic Integration of
New Words.'' \emph{Journal of Experimental Psychology: Learning, Memory,
and Cognition} 48 (7): 1064--81.
\url{https://doi.org/10.1037/xlm0001122}.

\leavevmode\hypertarget{ref-savic_experience_2023}{}%
---------. 2023. ``Experience and Maturation: The Contribution of
Co-Occurrence Regularities in Language to the Development of Semantic
Organization.'' \emph{Child Development} 94 (1): 142--58.
\url{https://doi.org/10.1111/cdev.13844}.

\leavevmode\hypertarget{ref-sedivy_pragmatic_2003}{}%
Sedivy, Julie C. 2003. ``Pragmatic Versus Form-Based Accounts of
Referential Contrast: Evidence for Effects of Informativity
Expectations.'' \emph{Journal of Psycholinguistic Research} 32 (1):
3--23.

\leavevmode\hypertarget{ref-sedivy_achieving_1999}{}%
Sedivy, Julie C., Michael K. Tanenhaus, Craig G. Chambers, and Gregory
N. Carlson. 1999. ``Achieving Incremental Semantic Interpretation
Through Contextual Representation.'' \emph{Cognition} 71 (2): 109--47.

\leavevmode\hypertarget{ref-sloutsky2004}{}%
Sloutsky, Vladimir M, and Anna V Fisher. 2004. ``Induction and
Categorization in Young Children: A Similarity-Based Model.''
\emph{Journal of Experimental Psychology: General} 133 (2): 166.

\leavevmode\hypertarget{ref-snow1972}{}%
Snow, Catherine E. 1972. ``Mothers' Speech to Children Learning
Language.'' \emph{Child Development}, 549--65.

\leavevmode\hypertarget{ref-sperber1986relevance}{}%
Sperber, Dan, and Deirdre Wilson. 1986. \emph{Relevance: Communication
and Cognition}. Vol. 142. Citeseer.

\leavevmode\hypertarget{ref-stahl2015}{}%
Stahl, Aimee E, and Lisa Feigenson. 2015. ``Observing the Unexpected
Enhances Infants' Learning and Exploration.'' \emph{Science} 348 (6230):
91--94.

\leavevmode\hypertarget{ref-tessler_informational_2020}{}%
Tessler, Michael Henry, Polina Tsvilodub, Jesse Snedeker, and Roger P.
Levy. 2020. ``Informational Goals, Sentence Structure, and Comparison
Class Inference.'' \emph{Proceedings of the Annual Conference of the
Cognitive Science Society}, January.
\url{https://par.nsf.gov/biblio/10159025-informational-goals-sentence-structure-comparison-class-inference}.

\leavevmode\hypertarget{ref-unger_statistical_2020}{}%
Unger, Layla, Olivera Savic, and Vladimir M. Sloutsky. 2020.
``Statistical Regularities Shape Semantic Organization Throughout
Development.'' \emph{Cognition} 198 (May): 104190.
\url{https://doi.org/10.1016/j.cognition.2020.104190}.

\leavevmode\hypertarget{ref-westerbeek2015}{}%
Westerbeek, Hans, Ruud Koolen, and Alfons Maes. 2015. ``Stored Object
Knowledge and the Production of Referring Expressions: The Case of Color
Typicality.'' \emph{Frontiers in Psychology} 6.

\leavevmode\hypertarget{ref-willits2008}{}%
Willits, Jon A, Rachel Shirley Sussman, and Michael S Amato. 2008.
``Event Knowledge Vs. Verb Knowledge.'' In \emph{Proceedings of the 30th
Annual Conference of the Cognitive Science Society}, 2227--32.

\leavevmode\hypertarget{ref-xiang_pragmatic_2022}{}%
Xiang, Ming, Christopher Kennedy, Weijie Xu, and Timothy Leffel. 2022.
``Pragmatic Reasoning and Semantic Convention: A Case Study on Gradable
Adjectives.'' \emph{Semantics and Pragmatics} 15 (September):
9:EA--9:EA. \url{https://doi.org/10.3765/sp.15.9}.

\leavevmode\hypertarget{ref-xu2007}{}%
Xu, Fei, and Joshua B Tenenbaum. 2007. ``Word Learning as Bayesian
Inference.'' \emph{Psychological Review} 114 (2): 245.

\leavevmode\hypertarget{ref-yoon2020}{}%
Yoon, Erica J, Michael Henry Tessler, Noah D Goodman, and Michael C
Frank. 2020. ``Polite Speech Emerges from Competing Social Goals.''
\emph{Open Mind} 4: 71--87.

\leavevmode\hypertarget{ref-yu2007}{}%
Yu, Chen, and Linda B Smith. 2007. ``Rapid Word Learning Under
Uncertainty via Cross-Situational Statistics.'' \emph{Psychological
Science} 18 (5): 414--20.
\end{cslreferences}
%\chapter{Introduction}
% Introductory stuff

%\chapter{A Chapter}
%\section{Introduction}
% Intro to chapter one

% Format a LaTeX bibliography
\makebibliography

% Figures and tables, if you decide to leave them to the end
%\input{figure}
%\input{table}

\end{document}
